\pdfminorversion=7%
\documentclass[aspectratio=169,mathserif,notheorems]{beamer}%
%
\xdef\bookbaseDir{../../bookbase}%
\xdef\sharedDir{../../shared}%
\RequirePackage{\bookbaseDir/styles/slides}%
\RequirePackage{\sharedDir/styles/styles}%
\toggleToGerman%
%
\definecolor{liuhui-r-color}{HTML}{CC6666}%
\def\liuhuir{\ensuremath{{\color{liuhui-r-color}r}}}%
\definecolor{liuhui-s6-color}{HTML}{CC0000}%
\def\liuhuiss{\ensuremath{{\color{liuhui-s6-color}s_6}}}%
\definecolor{liuhui-s12-color}{HTML}{0000CC}%
\def\liuhuist{\ensuremath{{\color{liuhui-s12-color}s_{12}}}}%
\definecolor{liuhui-y-color}{HTML}{80A000}%
\def\liuhuiy{\ensuremath{{\color{liuhui-y-color}y}}}%
\definecolor{liuhui-x-color}{HTML}{00A0A0}%
\def\liuhuix{\ensuremath{{\color{liuhui-x-color}x}}}%
\definecolor{liuhui-s24-color}{HTML}{00A000}%
\def\liuhuistf{\ensuremath{{\color{liuhui-s24-color}s_{24}}}}%
%
\subtitle{14.~Zwischenspiel:~Fehler im Kode mit Exceptions und IDE finden}%
%
\begin{document}%
%
\startPresentation%
%
\section{Einleitung}%
%%
\begin{frame}%
\frametitle{Einleitung}%
\begin{itemize}%
\item In den zukünftigen Slides werden wir eher weniger Screenshots vom \pycharm\ \glsFull{ide} verwenden.%
\item<2-> Stattdessen werden wir die Programme und ihre Ausgaben als Listings präsentieren.%
\item<3-> Das ist äquivalent mit dem Vorteil, dass wir Kode auch in den Slides selektieren und kopieren können.%
\item<4-> Bevor wir uns aber von den Screenshots verabschieden, wollen wir uns noch eine wichtige Funktion anschauen, die viele \pglspl{ide} bereitstellen\only<-4>{.}\uncover<5->{:}%
\item<5-> Sie können uns beim Suchen von Fehlern helfen.%
\end{itemize}%
\end{frame}%
%
\begin{frame}%
\frametitle{Fehler}%
\begin{itemize}%
\item Fehler sind häufig.%
\item<2-> \alert{Alle Programmierer machen Fehler.}%
\item<3-> Manchmal sind es Tippfehler.\uncover<4->{ %
Manchmal kommen wir der Reihenfolge von Parametern einer Funktion durcheinander.\uncover<5->{ %
Manchmal verwechseln wir einen \pythonil{int} mit einem \pythonil{float}.\uncover<6->{ %
Manchmal machen wir einfach logische Fehler.\uncover<7->{ %
Manchmal haben wir vielleicht die Funktion, die wir aufrufen, falsch verstanden.\uncover<8->{ %
Manchmal haben wir vielleicht den Algorithmus, den wir gerade versuchen zu implementieren, falsch verstanden.%
}}}}}%
\item<9-> Manche Fehler sind einfach zu finden.%
\item<10-> Manchmal führen wir ein Programm aus, as crashed, und die Ausgabe verrät uns, wo der Fehler ist.
\item<11-> Manchmal müssen wir lange \glslink{debugger}{debuggen}, um die Fehler zu entdecken~(das kommt später mal).%
\item<12-> Bei manchen einfachen Fällen, wie Tippfehlern, kann uns eine gute \pgls{ide} aber schon helfen.%
\end{itemize}%
\end{frame}%
%
\section{Fehler mit Hilfe der Ausgabe des Programms finden}%
%
\begin{frame}[t]%
\frametitle{Beispiel}%
%
\gitLoadAndExecPythonAndErrors{variables:assignment_wrong}{}{variables}{assignment_wrong.py}{}%
%
\begin{itemize}%
%
\only<-2>{%
\item Das Programm \textil{assignment_wrong.py}, eine fehlerhafte Variante des Programmes \textil{assignment.py} aus der vorigen Einheit.%
}%
%
\only<-3>{%
\item<2-> Können Sie den Fehler finden?%
}%
%
\item<3-> Genau. In Zeile~12 hat der Programmierer aus Versehen \pythonil{intvar} statt \pythonil{int_var} geschrieben.%
%
\end{itemize}%
\listingPython{}{variables:assignment_wrong}{0.15}{0.31}{0.7}{0.7}%
\end{frame}%
%
\begin{frame}[t]%
\frametitle{Fehler in der Programmausgabe}%
\begin{itemize}%
%
\only<-2>{%
\item Wenn wir das Programm ausführen, so bekommen wir einen Fehler angezeigt.%
}%
%
\only<-3>{%
\item<2-> Wir können das Programm auch in dem \pycharm\ \gls{ide} ausführen, in dem wir entweder auf \pycharmRun\ klicken oder in dem wir \keys{\shift+F10} drücken.%
}%
%
\only<-6>{%
\item<3-> Die Ausgabe stimmt genau mit dem Listing on vorhin überein: Etwas ist schief gegangen!%
}%
%
\only<-7>{%
\item<4-> Wenn ein Programm carashed, dann ist \alert{das Erste}, was wir machen, uns alle Ausgaben \alert{sehr genau} anzuschauen.%
}%
%
\only<-8>{%
\item<5-> In unserem Beispiel hier sagt uns der Text sogar ziemlich genau, was schief gegangen ist \emph{und} schlägt sogar vor, wie wir das korrigieren können.%
}%
%
\only<-9>{%
\item<6-> Da steht:~\emph{\inQuotes{\pythonilIdx{NameError}: name \inSQuotes{\pythonil{intvar}} is not defined. Did you mean: \inSQuotes{\pythonil{int_var}}?}}%
}%
%
\only<-10>{%
\item<7-> Das ist ziemlich klar.%
}%
%
\only<-11>{%
\item<8-> Wir haben auf eine Variable~(einen Namen) \pythonil{intvar} zugegriffen, die/den wir nicht definiert oder zugewiesen haben.%
}%
%
\only<-12>{%
\item<9-> Diese Variable existiert nicht.%
}%
%
\only<-13>{%
\item<10-> Der \python-Interpreter hat dann nachgeschlagen, ob es nicht eine Variable mit einem ähnlichen Namen gibt.%
}%
%
\only<-14>{%
\item<11-> Er hat gefunden, dass es eine Variable namens \pythonil{int_var} gibt.%
}%
%
\only<-15>{%
\item<12-> Mehrnoch, er teilt uns sogar die genaue Datei und Zeilennummer mit, wo der Fehler passiert ist:~in Zeile~12 von Datei~\textil{assignment_wrong.py}!%
}%
%
\only<-16>{%
\item<13-> Mit dieser Information haben wir gute Chancen, den Fehler zu finden und zu korrigieren.%
}%
%
\item<14-> Dieser so genannte \pythonil{Exception} \glslink{stackTrace}{Stack Trace}, der auf dem \emph{\glsFull{stderr}} ausgegeben wrd, sagt uns also nicht nur, dass es einen Fehler gab.%
 %
\item<15-> Er sagt uns auch, was der wahrscheinlichste Grund für den Fehler und wo er wahrscheinlich stattfand.%
%
\item<16-> Wir diskutieren das Thema \pythonilsIdx{Exception} später, aber bereits jetzt sollte die Sachse ziemlich klar sein:%
%
\end{itemize}%
%
\listingOutput{1,4-16}{variables:assignment_wrong}{}{0.1}{0.56}{0.9}{0.7}%
\locateGraphicTB{2}{width=0.8\paperwidth}{graphics/errorsInIde01runProgram}{0.1}{0.35}%
\locateGraphicTB{3}{width=0.8\paperwidth}{graphics/errorsInIde02exception}{0.1}{0.35}%
%
\uncover<17->{%
\bestPractice{readErrorMessage}{Always carefully \emph{read} error messages. %
They often provide you very crucial information where to look for the mistake. %
Not reading error messages is wrong.%
}}%%
%
\end{frame}%
%
\section{Fehler mit dem IDE suchen}%
%
\begin{frame}[t]%
\frametitle{Fehlersuche mit PyCharm}%
%
\begin{itemize}%
%
\only<-2>{%
\item Wir haben den Fehler gefunden, in dem wir das Programm ausgeführt haben und dann den Output gelesen haben.%
}%
%
\only<-3>{%
\item<2-> In der Konsole unten in \pycharm, in der die Ausgabe des Programms steht, können wir sogar auf die verlinkte Zeile in der Datei klicken.%
}%
%
\only<-4>{%
\item<3-> Das bringt uns direkt zu der Zeile mit dem Fehler.%
}%
%
\only<-5>{%
\item<4-> Die Frage ist: Hätten wir den Fehler auch finden können, \alert{ohne} das Programm auszuführen?%
}%
%
\only<-6>{%
\item<5-> Wenn wir uns die Kodezeile genau angucken, dann stellen wir fest, dass das falsch geschriebene eigentlich schon die ganze Zeit mit roter Farbe unterstrichen war!%
}%
%
\only<-8>{%
\item<6-> Das sollte uns bereits gesagt haben, dass hier irgendetwas nicht stimmt.%
}%
%
\only<8->{%
\only<-9>{%
\item<8-> Wir kennen jetzt also schon zwei Methoden, Fehler im Kode mit Hilfe des \pgls{ide} zu finden.%
}%
%
\only<-10>{%
\item<9-> Aber es gibt noch mehr.%
}%
%
\only<-11>{%
\item<10-> Die \pycharm\ \pgls{ide} zeigt uns auch das kleine rote~\pycharmErrorsSymbol~Symbol in der oberen rechten Ecke.%
}%
%
\only<-12>{%
\item<11-> Wenn wir darauf klicken, bekommen wir eine Liste möglicher Fehler und Warnungen angezeigt.%
}%
%
\only<-13>{%
\item<12-> Hier sagt uns \pycharm, dass es eine \emph{\inQuotes{Unresolved reference \inSQuotes{intvar}}} in Zeile~12 gibt.%
}%
%
\only<-14>{%
\item<13-> Klicken wir auf diese Nachricht, dann bringt uns das wieder zu der fehlerhaften Zeile.%
}%
%
\only<-15>{%
\item<14-> Zusätzlich gibt es auch kleine rote Markierungen am rechten Rand des Editorfensters.%
}%
%
\only<-15>{%
\item<15-> Halten wir den Mauskursor über die Markierung, dann öffnet sich eine kleine Ansicht mit der entsprechenden Warnung.%
}%
%
\only<-17>{%
\item<16-> Wir können auch auf den kleinen \pycharmErrorsButton-Button im Seitenmenü auf der linken Seite klicken oder \keys{\Alt+6} drücken.%
}%
%
\only<-18>{%
\item<17-> Das bringt uns dann wieder zu der Liste mit möglichen Fehlern.%
}%
}%
\end{itemize}%
%
\locateGraphicTB{1-2}{width=0.8\paperwidth}{graphics/errorsInIde02exception}{0.1}{0.35}%
\locateGraphicTB{3-6,8-9}{width=0.8\paperwidth}{graphics/errorsInIde03underlined}{0.1}{0.35}%
\locateGraphicTB{10-11}{width=0.8\paperwidth}{graphics/errorsInIde04errors}{0.1}{0.35}%
\locateGraphicTB{12}{width=0.8\paperwidth}{graphics/errorsInIde05errorsList}{0.1}{0.35}%
\locateGraphicTB{13}{width=0.8\paperwidth}{graphics/errorsInIde06errorsListToLine}{0.1}{0.35}%
\locateGraphicTB{14-15}{width=0.8\paperwidth}{graphics/errorsInIde07errorMark}{0.1}{0.35}%
\locateGraphicTB{16}{width=0.8\paperwidth}{graphics/errorsInIde08sidebar}{0.1}{0.35}%
\locateGraphicTB{17}{width=0.8\paperwidth}{graphics/errorsInIde09sidebarToView}{0.05}{0.35}%%
%
\only<7>{%
\bestPractice{redUnderline}{%
Wenn wir Kode schreiben, sollten wir immer aufpassen, ob das \pgls{ide} uns über irgendwelche mögliche Fehler benachrichtigt. %
Bei \pycharm\ bekommen wir solche Benachrichtigungen oftmals durch rote oder gelbe Unterstreichungen. %
Wir sollten solche Markierungen immer überprüfen.%
}%
}%
%
\only<18->{%
\usefulTool{ideForErrors}{%
Das \pgls{ide} und die Fehlermeldungen (\pythonilIdx{Exception} \glslink{stackTrace}{Stack Traces}) sind Ihre wichtigsten Werkzeuge um Fehler zu finden. %
Lesen Sie die Fehlermeldungen. %
Gleichgültig ob Sie \pycharm\ oder irgendein anderes \pgls{ide} verwenden, wenn dieses \pgls{ide} Ihren Kode mit Warnungen und Fehlerhinweisen annotiert, dann lesen und prüfen Sie jeden einzelnen Hinweis.%
}%
}%
%
\end{frame}%
%
\section{Zusammenfassung}%
%
\begin{frame}%
\frametitle{Fehler}%
\begin{itemize}%
\item Programmierer machen Fehler.%
\item<2-> Wir machen Fehler.%
\item<3-> Sie werden viele Fehler machen.%
\item<4-> Jetzt und später.%
\item<5-> Selbst in einfachsten Programmen werden wir Fehler machen.%
\item<6-> Das kann nicht verhindert werden.%
\item<7-> Die Frage ist also, wie wir damit umgehen.%
\item<8-> Wie können wir die Anzahl der Fehler, die wir machen werden, minimieren?%
\item<9-> Wie können wir die Anzahl der Fehler, die wir finden und korrigieren, maximieren=?%
\item<10-> Die Antwort ist:~\alert{Mit allen zur Verfügung stehenden Mitteln}.%
\end{itemize}%
\end{frame}%
%
\begin{frame}%
\frametitle{Werkzeuge}%
\begin{itemize}%
\item Wir nutzen \alert{alle} Werkzeuge, die wir finden können, um Fehler zu entdecken und zu korrigieren.%
\item<2-> Wenn wir ein Programm ausführen und es crashed, dann lesen wir Ausgabe genau.%
\item<3-> Wenn unser \pgls{ide} mögliche Fehler und Warnungen anzeigt, dann schauen wir uns diese genau an.%
\item<4-> Beides kann unglaublich hilfreich sein.%
\item<5-> Randnotiz:~Sie können an der Anzahl der verschiedenen Methodn, wie \pycharm\ Warnhinweise präsentiert, erkennen, wie wichtig das ist.%
\end{itemize}%
\end{frame}%
%
\begin{frame}%
\frametitle{Unentdeckte Fehler}%
\begin{itemize}%
\item Selbst wenn Ihr Programm wie erwartet funktioniert, dann können trotzdem versteckte Fehler irgendwo im Kode sein.%
%
\item<2-> Manchmal kann man einfach festellen, ob die Ausgabe eines Programms korrect ist.%
%
\item<3-> Manchmal kann man das nicht.%
%
\item<4-> Manchmal kann ein Output, der OK aussieht, trotzdem falsch sein.%
%
\item<5-> Manchmal können fehlerhafte Befehle in einem Programm sein, die nur bei der aktuellen Ausführung nicht erreicht urden.%
%
\item<6-> Selbst korrekte Ausgaben garantieren uns nicht, dass unsere Programme korrekt sind.%
%
\item<7-> Prüfen Sie daher immer alle Hinweise Ihres \pgls{ide}.%
%
\item<8-> Stellen Sie sicher, dass Sie immer alle Fehler- und Warnnachrichten vollständig verstanden haben.%
%
\end{itemize}%
\end{frame}%
%
\begin{frame}%
\frametitle{Warnhinweise}%
%
\begin{itemize}%
\item Warnhinweise können ebenfalls wichtig sein.%
\item<2-> Sie können uns auf mögliche Problem hinweise.%
\item<3-> Vielleicht hat eine Variable einen falschen Typ.%
\item<4-> Vielleicht fehlt ein benötigtes Paket.%
\item<5-> Korrigieren Sie alle Fahler und Warnungen wo immer möglich.%
\item<6-> Selbst wenn Sie denken, dass es False-Positives sind, schauen Sie, ob Sie sie nicht doch korrigieren können.%
\item<7-> Denn False-Positives könnten auch von falsch formatiertem Kode herrühren, der für andere Entwickler schwer zu verstehen.%
\item<8-> Versuchen Sie immer Warnungs- und Fehler-freien Kode zu produzieren.%
\item<9-> Denn, auf der einen Seite, könnten Sie ja falsch liegen, selbst wenn Sie denken, dass die Warnung falsch und der Kode korrekt ist\dots%
\item<10-> Auf der anderen Seite wird es einfacher, die echten Fehler zu finden, wenn es weniger Warnungen gibt.%
\end{itemize}%
\end{frame}%
%
\begin{frame}%
\frametitle{Zusammenfassung}%
\begin{itemize}%
\item Wir haben nun zwei wichtige Methoden kennengelernt, Fehler in unserem Kode zu finden.%
\item<2-> Auf der einen Seite gibt uns die Ausgabe bei Programmabstürzen schon ziemlich viele Informationen.%
\item<3-> Auf der anderen Seite können wir viele Hinweise und Warnungen bereits in dem \pycharm\ \pgls{ide} sehen.%
\item<4-> Wichtig ist, dass wir immer allen Warnungen und Fehlermeldungen nachgehen.%
\item<5-> Wir können nicht verhindern, dass wir Fehler machen.%
\item<6-> Aber wir sollten es uns möglichst einfach machen, Fehler zu finden.%
\end{itemize}%
\end{frame}%
%
\endPresentation%
\end{document}%%
\endinput%
%

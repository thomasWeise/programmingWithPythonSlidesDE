\pdfminorversion=7%
\documentclass[aspectratio=169,mathserif,notheorems]{beamer}%
%
\xdef\bookbaseDir{../../bookbase}%
\xdef\sharedDir{../../shared}%
\RequirePackage{\bookbaseDir/styles/slides}%
\RequirePackage{\sharedDir/styles/styles}%
\toggleToGerman%
%
\subtitle{6.~Beispiele Herunterladen}%
%
\begin{document}%
%
\startPresentation%
%
\section{Einleitung}%
%
\begin{frame}[t]%
\frametitle{Einleitung}%
\begin{itemize}%
\item Im Rest dieses Kurses werden wir intensiv mit praktischen Beispielen arbeiten.%
\item<2-> Wenn wir ein Konzept vorstellen, dann probieren wir das praktisch aus.%
\item<3-> Damit Sie diese Beispiele nachvollziehen können, stellen wir sie in einem \github\ repository zur Verfügung.%
\item<4-> Dieses Repository finden Sie unter \expandafter\url{\programmingWithPythonCodeRepo}.%
\end{itemize}%
\locateGraphicTB{}{width=0.4\paperwidth}{graphics/downloadExamples}{0.3}{0.46}%
\end{frame}%
%
\section{Herunterladen der Beispiele}%
%
\begin{frame}[t]%
\frametitle{Herunterladen der Beispiele}%
\begin{itemize}%
\only<-2>{\item Öffnen Sie einen Webbrowser und besuchen Sie die Webseite \expandafter\url{\programmingWithPythonCodeRepo}.%
\only<2>{ Klicken Sie auf das nach unten gerichtete Dreieck im Button \menu{Code}.}%
}%
%
\only<3>{\item<3-> In dem sich öffnenden Menü, klicken Sie auf \menu{Download ZIP}.}%
%
\only<4-5>{\item<4-> Der Download beginnt und ist irgendwann abgeschlossen.%
\only<5->{ Öffnen Sie die heruntegrladene Datei.}}%
\only<6->{\item<6-> Die Datei ist ein ZIP-Archiv, also eine Datei, die andere Dateien und Verzeichnisse enthält.%
\only<7->{ Sie können Sie an einem Ihnen angenehmen Ort entpacken. %
\only<8->{ Das erste große Beispiel, mit dem wir als nächstes beginnen, befindet sich üprigens im Unterordner \textil{veryFirstProject}.}%
}%
}%
\end{itemize}%
%
\locateGraphicTB{-2}{width=0.6\paperwidth}{graphics/downloadingExamples1website}{0.2}{0.33}%
\locateGraphicTB{3}{width=0.6\paperwidth}{graphics/downloadingExamples2download}{0.2}{0.33}%
\locateGraphicTB{4}{width=0.6\paperwidth}{graphics/downloadingExamples3downloaded}{0.2}{0.33}%
\locateGraphicTB{5}{width=0.6\paperwidth}{graphics/downloadingExamples4open}{0.2}{0.33}%
\locateGraphicTB{6-7}{width=0.6\paperwidth}{graphics/downloadingExamples5zipOpened}{0.2}{0.33}%
\locateGraphicTB{8-}{width=0.6\paperwidth}{graphics/downloadingExamples6veryFirstProject}{0.2}{0.33}%
\end{frame}%
%
\section{Repository Klonen}%
%
\begin{frame}[t]%
\frametitle{Repository auf GitHub}%
\begin{itemize}%
\item Alternativ zum direkten Herunterladen des \texttt{zip}-Archivs können Sie das Repository mit den Beispielen für diesen Kurs auch einfach in \pycharm\ \emph{clonen}.%
\item<2-> Unsere Beispiele befinden sich nämlich in einem sogenannten \git-Repository\cite{S2023LG,T2024BGAGVCPMATFTND}.%
\item<3-> \git\ ist ein \glsreset{VCS}\pgls{VCS}\cite{S2023LG,T2024BGAGVCPMATFTND}, also ein Versionsmanagmentsystem für Softwareentwicklung.%
\item<4-> Mit so einem System können wir schrittweise an unserem Kode arbeiten und Änderungen in die Kodebasis einfügen.%
\item<5-> Das \pgls{VCS} merkt sich die Geschichte unseres Projekts und erlaubt es uns, kollaborativ gemeinsam an dem Kode zu arbeiten.%
\item<6-> Also wir werden nicht kollaborativ gemeinsam an dem Beispiel-Kode arbeiten {\dots} es sind ja Beispiele für diesen Kurs.%
\item<7-> Aber \git\ ist ein sehr weit verbreites \pgls{VCS}, also schaded es nichts, zumindest grob zu wissen, wie das funktioniert.%
\item<8-> Unsere Beispiele liegen in einem \git-Repository auf \github\cite{PRGWSUdVLFTEKPKFBV2016TSRFTAOGAG,T2024BGAGVCPMATFTND}.%
\item<9-> Clonen wir dieses Repository also!%
\end{itemize}%
\end{frame}%
%
\begin{frame}[t]%
\frametitle{Repository Klonen}%
\begin{itemize}%
\only<-1>{\item Clonen wir dieses Repository also!}%
%
\only<-2>{\item<2-> Im \pycharm\ Willkommensbildschirm, klicken Sie auf \menu{Clone Repository}.}%
%
\only<-3>{\item<3-> Geben Sie \expandafter\url{\programmingWithPythonCodeRepo} als \menu{URL:} ein.}%
\only<-5>{\item<4-> Wählen Sie ein Verzeichnis als \menu{Directory:} aus, wo das neue Projekt mit den Beispielen angelegt werden soll.\uncover<5->{ Nehmen Sie nicht \bashil{/tmp}.}}%
\only<-6>{\item<6-> Klicken Sie auf \menu{Clone}.}%
%
\only<-7>{\item<7-> Der Klon-Prozess beginnt.}%
%
\only<-9>{\item<8-> Nach dem herunterladen wird \pycharm\ fragen, ob wir dem Projekt vertrauen.\uncover<9->{ Wenn ja, dann klicken Sie auf \menu{Trust Project}.}}%
%
\item<10-> Nun können wir alle Beispieldateien sehen.%
\end{itemize}%
%
\locateGraphicTB{2}{width=0.45\paperwidth}{graphics/clone01welcomeToPycharm.pdf}{0.275}{0.27}%
\locateGraphicTB{3-6}{width=0.45\paperwidth}{graphics/clone02selectRepoAndDestination.pdf}{0.275}{0.27}%
\locateGraphicTB{7}{width=0.45\paperwidth}{graphics/clone03cloning}{0.275}{0.25}%
\locateGraphicTB{8-9}{width=0.75\paperwidth}{graphics/clone04trustProject.pdf}{0.125}{0.3}%
\locateGraphicTB{10-}{width=0.72\paperwidth}{graphics/clone05finished}{0.14}{0.235}%
\end{frame}%
%
\section{Zusammenfassung}%
%
\begin{frame}%
\frametitle{Zusammenfassung}%
\begin{itemize}%
\item Wir haben nun das Repository mit den Beispielen für diesen Kurs heruntergeladen.%
\item<2-> Damit haben Sie alle Programme, die wir im folgenden verwenden, direkt zur Hand.%
\item<3-> Sie können also unsere Beispiele sehr komfortabel nachvollziehen.%
\item<4-> Als Seiteneffekt haben wir auch einen kurzen Blick auf \git\ und \github\ geworfen.%
\end{itemize}%
\end{frame}%
%
\endPresentation%
\end{document}%%
\endinput%
%

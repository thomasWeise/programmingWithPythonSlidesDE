\pdfminorversion=7%
\documentclass[aspectratio=169,mathserif,notheorems]{beamer}%
%
\xdef\bookbaseDir{../../bookbase}%
\xdef\sharedDir{../../shared}%
\RequirePackage{\bookbaseDir/styles/slides}%
\RequirePackage{\sharedDir/styles/styles}%
\toggleToGerman%
%
\subtitle{22.~Alternativen~mit~\texttt{if}}%
%
\begin{document}%
%
\startPresentation%
%
\section{Einleitung}%
%%
\begin{frame}[t]%
\frametitle{Einleitung}%
%
\begin{itemize}%
\item Mit den Dingen, die wir bisher gelernt haben, können wir einfache lineare Programme implementieren.%
\item<2-> Diese Programme führen Befehl nach Befehl aus, genau in der Reihenfolge, in der wie diese aufgeschrieben haben.%
\item<3-> Wenn wir einen Befehl zweimal ausführen wollen, dann müssen wir ihn zweimal hinschreiben.%
\item<4-> Unsere Programme können ihr Verhalten nicht basierend auf den Eingabedaten ändern.%
\item<5-> Sie können nicht Befehl~$\mathcal{A}$ ausführen, wenn eine Variable einen bestimmten Wert hat, und sonst Befehl~$\mathcal{B}$.%
\item<6-> Sie können weder verzweigen noch Schritte wiederholen.%
\item<7-> In den folgenden Einheiten werden wir lernen, wie man verzweigt und wiederholt.%
\item<8-> Wir lernen Befehle, die den Kontrollfluss ändern.%
%
\end{itemize}%
\end{frame}%
%
\begin{frame}%
\frametitle{Alternativen}%
\begin{itemize}%
\item Alternativen erlauben uns, ein Stück Kode \emph{nur dann} auszuführen, wenn ein bestimmter Boolescher Ausdruck \pythonilIdx{True} ergibt.%
%
\item<2-> Somit können unsere Programme Entscheidungen treffen.%
%
\item<3-> Sie können einen Befehl~$\mathcal{A}$ in einer Situation ausführen und einen Befehl~$\mathcal{B}$ in einer anderen Situation.%
%
\item<4-> Alternativen sind somit die fundamentalsten Kontrollflussstatements.%
\end{itemize}%
\end{frame}%
%
\section{Einfache Alternative mit \texttt{if}}%
%
\begin{frame}[t]%
\frametitle{Einfache Alternative mit \texttt{if}}%
%
\begin{itemize}%
\only<-3>{%
\item Eine einfache Alternative ist ein Stück Programkode das eine Menge von Befehlen ausführt wenn eine Kondition zutrifft.%
\item<2-> Die Kondition ist ein einfacher Boolescher Ausdruck, wie wir ihn bereits in Einheit~\unitBool~(\pythonil{bool}) diskutiert haben.%
\item<3-> Die Syntax dafür ist sehr einfach\cite{PSF:P3D:TPT:MCFT}:%
}%
%
\only<-8>{%
\item<4-> Die erste Zeile beginnt mit~\pythonilIdx{if} gefolgt von einem Booleschen Ausdruck, gefolgt von einem Doppelpunkt~(\pythonilIdx{:}).%
}%
%
\item<5-> Nur wenn der Boolesche Ausdruck \pythonil{True} ergibt, dann wird der \emph{eingerückte} Block von Kode unter dem \pythonil{if} ausgeführt.%
%
\only<7->{%
\item<7-> \alert{Jede} konditionelle Zeile Kode des \pythonil{if}-Statements ist mit vier Leerzeichen eingerückt.%
%
\item<8-> Nach dem eingerückten Block folgt normaler Programmkode.%
%
\item<9-> Dieser Kode ist nicht eingerückt und wird ausgeführt, egal was der Boolesche Ausdruck im \pythonil{if} ergeben hat.
}%
%
\end{itemize}%
%
\uncover<6>{%
\bestPractice{indentation}{Blöcke von Kode werden mit vier Leerzeichen eingerückt\cite{PEP8}.}%
}%
%
\gitLoadPython{syntax/if}{}{syntax/if.py}{}%
\listingPythonSyntax{3-5,7-}{syntax/if}{0.05}{0.5}{0.9}{0.6}%
\end{frame}%
%
\begin{frame}%
\frametitle{Beispiel}%
%
\parbox{0.4\paperwidth}{\small{%%
\begin{itemize}%
\only<-6>{%
\item Benutzen wir doch mal ein \pythonil{if} um zu prüfen, ob ein Jahr ein Schaltjahr ist.%
}%
%
\only<-7>{%
\item<2-> Nach dem Gregorianischen Kalender ist ein Jahr~$y$ ein Schaltjahr wenn $y$~durch~4 geteilt werden kann, aber nicht durch~100; oder wenn es durch~400~geteilt werden kann\cite{EOEBEB:LR}.%
}%
%
\only<-8>{%
\item<3-> Zuerst speichern wir das Jahr, was wir untersuchen wollen, in einer Variable~\pythonil{year}.%
}%
\only<-4>.%
}%
%
\only<-9>{%%
\item<5-> Wenn wir den \glslink{modulodiv}{Modulo}-Operator auf zwei Werte \pythonil{a} und \pythonil{b} anwenden, liefert er uns den Rest der Division~\pythonil{a // b}\pythonIdx{//}.%
}%
%
\only<-10>{%
\item<6-> Wenn der \glslink{modulodiv}{Rest}~0~ist, dann ist \pythonil{a} durch \pythonil{b} teilbar.%
}%
%
\only<-10>{%
\item<7-> Zum Beispiel \pythonil{10 \% 5 == 0} aber \pythonil{10 \% 6 == 4} und \pythonil{10 \% 3 == 1}.%
}%
%
\only<-11>{%
\item<8-> Um zu prüfen, ob \pythonil{year} durch~4 teilbar ist, brauchen wir nur zu prüfen ob~\pythonil{(year \% 4) == 0}.%
}%
%
\only<-12>{%
\item<9-> \pythonil{year} soll nicht durch~100 teilbar sein, also prüfen wir auch~\pythonil{(year \% 100) != 0}~(ein Rest ungleich~0 bedeutet \inQuotes{nicht teilbar}).%
}%
%
\only<-13>{%
\item<10-> Wir müssen jetzt beide Bedingungen mit~\pythonilIdx{and} kombinieren und schreiben~\pythonil{((year \% 4) == 0)}\linebreak[2]\pythonil{ and ((year \% 100) != 0)}.%
}%
%
\only<-14>{%
\item<11-> Das Ergebnis ist nur dann \pythonil{True}, wenn beide Bedingungen \pythonil{True} sind, also wenn das Jahr durch~4 aber nicht durch~100 teilbar ist.%
}%
%
\only<-15>{%
\item<12-> Wir benutzen Klammern um de Ausdruck einfacher lesbar zu machen.%
}%
%
\only<-16>{%
\item<13-> Wenn diese kombinierte Bedingung \pythonilIdx{False} ist, dann könnte das Jahr immer noch ein Schaltjahr sein, nämlich wenn es durch~400 teilbar ist.%
}%
%
\only<-17>{%
\item<14-> Die letzte Bedingung ist also, zu prüfen ob~\pythonil{(year \% 400) == 0}.%
}%
%
\only<-18>{%
\item<15-> Wir fügen diese Bedingung mit \pythonil{or} zu unserem Ausdruck hinzu, weil es ausreicht, wenn entweder unser erster Teilausdruck \emph{oder} diese Bedigung wahr ist.%
}%
%
\only<-19>{%
\item<16-> Wir bekommen den Ausdruck \pythonil{(((year \% 4) == 0)}\linebreak[2]\pythonil{ and ((year \% 100) != 0))}\linebreak[2]\pythonil{ or ((year \% 400) == 0)}.%
}%
%
\only<-21>{%
\item<17-> Wir nutzen wieder Klammern um die Teilausdrücke schon für die Lesbarkeit zu gruppieren.%
}%
%
\only<-22>{%
\item<18-> Alles, was wir jetzt noch machen müssen, ist ein \pythonil{if} vor den Ausdruck und ein~\pythonil{:} dahinter zu schreiben, und schon haben wir eine Alternative konstruiert.%
}%
%
\only<-23>{%
\item<19-> Wenn der Ausdruck \pythonil{True} ergibt, dann drucken wir~\pythonil{f\"\{year\} is a leap year.\"}.
}%
%
\only<-24>{%
\item<20-> Dafür müssen wir das \pythonilIdx{print} Kommando mit vier Leerzeichen einrücken.%
}%
%
\item<21-> Wir kopieren diesen Kode zweimal und prüfen die Jahre~\pythonil{2024} und~\pythonil{1900}.%
%
\item<22-> Das zweite \pythonilIdx{if} beginnt mit der selben Einrückung wie das erste \pythonil{if}.%
%
\item<23-> Es wir ausgeführt, gleichgültit wie die erste Alternative ausgegangen ist.%
%
\item<24-> Am Ende geben wir noch \textil{End of year checking.} auf der Konsole aus.%
%
\item<25-> Diese letzte Zeile ist auch nicht eingerückt, wird also auf jeden Fall ausgeführt.%
%
\item<26-> Wir sehen, das 2024 ein Schaltjahr war, 1900~aber nicht.%
\end{itemize}%
}}%
%
\gitLoadAndExecPython{conditionals:if}{}{conditionals}{if_example.py}{}%
\listingPythonAndOutput{}{conditionals:if}{}{0.45}{0.1}{0.54}{0.92}%
\end{frame}%
%
\section{Kombinierte Alternative mit \texttt{if-else}}%
%
\begin{frame}[t]%
\frametitle{Kombinierte Alternative mit \texttt{if-else}}%
%
\begin{itemize}%
%
\only<-5>{%
\item Die Möglichkeit, eine Aktion auszuführen, wenn eine Bedingung wahr ist, ist schonmal gut.%
%
\item<2-> Oftmals wollen wir aber eine Aktion ausführen, wenn die Bedingung wahr ist und eine andere Aktion, wenn die Bedingung falsch ist.%
%
\item<3-> Technisch gesehen können wir das auch jetzt schon.%
%
\item<4-> Wir brauchen einfach nur zwei Alternativen mit der selben Bedingung~\pythonil{if} zu definieren, wobei wir bei der zweiten einfach \pythonil{not} vor die Bedigung schreiben.%
}%
%
\only<-9>{%
\item<5-> Natürlich geht das auch einfacher: mit \pythonil{if...else}\cite{PSF:P3D:TPT:MCFT}.%
}\only<-10>{%
\item<7-> Das \pythonil{if...else}-Statement beginnt genau wie ein normales \pythonilIdx{if}-Statement.%
}\only<-11>{%
\item<8-> \pythonil{if}, gefolgt von einem Booleschen Ausdruck, gefolgt vom Doppelpunkt~(\pythonilIdx{:}) stellt die Bedingung dar.%
}\only<-12>{%
\item<9-> Der folgende Block, wo jedes Statement mit vier Leerzeichen eingerückt ist, wird nur ausgeführt, wenn der Boolesche Ausdruck \pythonil{True} ergibt.%
}%
%
\item<10-> Dann folgt die \pythonil{else:} Zeile.%
%
\item<11-> Diese ist \emph{nicht} eingerückt und befindet sich in der selben Ebene wie das~\pythonil{if}.%
%
\item<12-> Danach kommt wieder ein ein mit vier Leerzeichen eingerückter Block.%
%
\item<13-> Dieser Block wird nur dann ausgeführt, wenn der Boolesche Ausdruck im~\pythonil{if} \pythonil{False} ergibt.%
%
\end{itemize}%
%
%
\gitLoadPython{syntax/if_else}{}{syntax/if_else.py}{}%
\listingPythonSyntax{6-}{syntax/if_else}{0.1}{0.5}{0.8}{0.6}%
\end{frame}%
%
\begin{frame}[t]%
\frametitle{Beispiel}%
%
\parbox{0.4\paperwidth}{\small{%%
\begin{itemize}%
%
\item Wir passen unser Schaltjahr-Program an, so dass es auch Text ausgibt, wenn das Jahr kein Schaltjahr ist.%
%
\item<2-> Das können wir mit dem neuen \pythonil{else}-Zweig machen.%
%
\item<3-> Wir zeigen auch, dass mehrere Statements sowohl im \pythonil{if}-\ als auch im \pythonil{else}-Zweig platziert werden können.%
\end{itemize}%
}}%
%
\gitLoadAndExecPython{conditionals:if_else}{}{conditionals}{if_else_example.py}{}%
\listingPythonAndOutput{}{conditionals:if_else}{}{0.45}{0.1}{0.54}{0.92}%
\end{frame}%
%
\begin{frame}%
\frametitle{Verschachteln von Alternativen}%
\begin{itemize}%
\item Wir können also mehrere Statements in die Blöcke der \pythonil{if}-\ und \pythonil{if...else}-Statements packen.%
%
\item<2-> Moment. \uncover<3->{\emph{\inQuotes{\pythonil{if}-\ und \pythonil{if...else}-Statements}} \dots\uncover<4->{ das sind auch Statements?}}%
%
\item<5-> Können wir also \pythonil{if}-\ und \pythonil{if...else}-Statements in die Kodeblöcke von \pythonil{if}-\ und \pythonil{if...else}-Statements packen?%
%
\item<6-> Können wir Alternativen verschachteln??%
%
\item<7-> Ja, können wir.%
\end{itemize}%
\end{frame}%
%
\begin{frame}[t]%
\frametitle{Beispiel}%
%
\parbox{0.4\paperwidth}{\small{%%
\begin{itemize}%
%
\only<-5>{%
\item Probieren wir das mal aus in unserem Programm \programUrl{conditionals:if_else_nested}.%
}\only<-6>{%
\item<2-> Wir haben drei Zahlen in den Variablen~\pythonil{a}, \pythonil{b}, and~\pythonil{c}.%
}\only<-7>{%
\item<3-> Wir wollen das Maximum dieser drei Zahlen herausfinden, also die größte Zahl, die entweder in~\pythonil{a}, \pythonil{b}, oder~\pythonil{c} gespeichert ist.%
}\only<-8>{%
\item<4-> Wir prüfen erstmal ob \pythonil{a > b}.%
}\only<-9>{%
\item<5-> Wenn das \pythonilIdx{True} ergibt, dann kann \pythonil{b} auf keinen Fall die größte Zahl sein. Nur \pythonil{a} oder \pythonil{c} kommen in Frage.%
}\only<-10>{%
\item<6-> Im Block des ersten \pythonilIdx{if}-Statements brauchen wir also nur noch zu prüfen, ob~\pythonil{a > c}.%
}\only<-10>{%
\item<7-> Beachten Sie, dass das zweite~\pythonilIdx{if}, der dazugehörige Block, das \pythonil{else}, und dessen Block vier Leerzeichen tiefer eingerückt sind, als das äußere~\pythonil{if}.%
}\only<-12>{%
\item<8-> Falls \pythonil{a > c} wahr ist, dann drucken wir den \glslink{fstring}{f-String} \pythonil{f\"\{a\} is the greatest number.\"} aus.%
}\only<-13>{%
\item<9-> Wenn es dagegen \pythonil{False} ist, dann gilt entweder \pythonil{a < c} oder \pythonil{a == c}.%
}\only<-14>{%
\item<10-> So oder so, \pythonil{f\"\{c\} is the greatest number.\"} ist richtig.%
}\only<-15>{%
\item<11-> Nun müssen wir wieder zum äußeren~\pythonil{if} zurück und den alternativen Zweig bearbeiten, den \pythonilIdx{else}-Teil.%
}\only<-16>{%
\item<12-> Was ist, wenn \pythonil{a > b} nicht \pythonilIdx{True} war?%
}\only<-17>{%
\item<13-> Dann gilt entweder \pythonil{b > a} oder \pythonil{b == a}.%
}\only<-18>{%
\item<14-> Es reicht dann, \pythonil{b} mit \pythonil{c} zu vergleichen.%
}\only<-19>{%
\item<15-> Wir rücken also ein weiteres \pythonilIdx{if} ein, diesmal mit der Bedingung \pythonil{b > c}.%
}\only<-20>{%
\item<16-> Wenn diese \pythonilIdx{True} ist, dann ist der größte Wert gleich dem in \pythonil{b} gespeicherten Wert.%
}\only<-20>{%
\item<17-> Wir können also \pythonil{f\"\{b\} is the greatest number.\"} drucken.%
}\only<-21>{%
\item<18-> Wenn die Bedingung dagegen \pythonilIdx{False} war, dann kommen wir in den \pythonilIdx{else}-Zweig.%
}%
\item<19-> Wir wissen, dass \pythonil{b >= a} und das entweder \pythonil{c > b} oder \pythonil{c >= b} gilt.%
%
\item<20-> Dafür können wir nun wieder \pythonil{f\"\{c\} is the greatest number.\"} drucken.%
%
\item<21-> Anmerkung: \python\ hat die beiden Funktionen \pythonil{min} und \pythonil{max}, die jeweils eine Sequenz von Werten als Parameter bekommen und den kleinsten bzw.\ größten Wert davon zurückliefern.%
%
\item<22-> In den letzten beiden Zeilen unseres Programms wenden wir sie auf die Variablen \pythonil{a}, \pythonil{b}, und~\pythonil{c} an.%
\end{itemize}%
}}%
%
\gitLoadAndExecPython{conditionals:if_else_nested}{}{conditionals}{if_else_nested.py}{}%
\listingPythonAndOutput{}{conditionals:if_else_nested}{}{0.45}{0.1}{0.54}{0.92}%
\end{frame}%
%
\section{Mehrfach Alternative mit \texttt{if-elif-else}}%
%
\begin{frame}[t]%
\frametitle{Mehrfach Alternative mit \texttt{if-elif-else}}%
%
\begin{itemize}%
\only<-6>{%
\item In manchen Fällen müssen wir Alternativen so verschachteln, dass der \pythonil{else}-Block des ersten~\pythonils{if} ein weiteres \pythonil{if} beinhaltet, und genau nur das~\pythonil{if}.%
\uncover<2->{%
\begin{itemize}%
\item Dessen \pythonil{else}-Block beinhaltet dann vielleicht ein weiteres \pythonil{if} beinhaltet, und genau nur dieses~\pythonil{if}.%
\uncover<3->{%
\begin{itemize}%
\item Dessen \pythonil{else}-Block beinhaltet dann vielleicht ein weiteres \pythonil{if} beinhaltet, und genau nur dieses~\pythonil{if}.%
\uncover<4->{ %
Und so weiter.%
}%
\end{itemize}%
}%
\end{itemize}%
}%
%
\item<5-> Das sieht ziemlich häßlich aus, weil wir ja jede neue Alternative wieder mit vier Leerzeichen einrücken müssen.%
%
\item<6-> Es füllt auch die Horizontale unseres Bildschirms schnell aus.%
}%
%
\item<7-> Darum wurde das \pythonilIdx{elif}-Statement entwickelt, das genau ein \pythonil{else}, welches nur ein \pythonil{if} beinhaltet, ersetzt~(oder eher kombiniert)\cite{PSF:P3D:TPT:MCFT}.%
%
\item<9-> Anstelle dem \pythonil{else} mit dem darin eingerückten \pythonil{if} schreiben wir einfach \pythonil{elif} mit der Bedingung, die das \pythonil{if} gehabt hätte.%
%
\item<10-> Wir können optional am Ende noch einen \pythonil{else}-Block haben, der nur dann ausgeführt wird, wenn \emph{keine} der Bedingungen des \pythonil{if} und der \pythonils{elif} wahr war.%
%
\end{itemize}%
%
\gitLoadPython{syntax/if_elif_else}{}{syntax/if_elif_else.py}{}%
\listingPythonSyntax{8-}{syntax/if_elif_else}{0.2}{0.49}{0.6}{0.6}%
\end{frame}%
%
\begin{frame}[t]%
\frametitle{Beispiel~(1)}%
%
\parbox{0.4\paperwidth}{\small{%%
\begin{itemize}%
%
\only<-5>{%
\item Wir benutzen diese Syntax\only<-4>{ in Programm \programUrl{conditionals:if_elif}}, um die Lebensphase einer Person basierend auf ihrem Alter zu berechnen.%
}\only<-6>{%
\item<2-> Wir definieren die Ganzzahl-Variable~\pythonil{age} für das Alter und setzen sie auf, sagen, wir~42.%
}\only<-6>{%
\item<3-> Wir wollen einen String in die Variable \pythonil{phase} schreiben, der die Lebensphase der Person beschreibt.%
}\only<-7>{%
\item<4-> Am Anfang steht nichts in der Variable, also setzen wir \pythonil{phase = None}.%
}\only<-8>{%
\item<5-> Wir benutzen den \glslink{typeHint}{Type Hint} \pythonil{str | None} um eine Variable zu spezifizieren, die entweder \pythonil{None} oder eine String beinhalten kann\cite{PEP604}.%
}\only<-9>{%
\item<6-> Wir beginning mit der ersten Bedingung und schreiben \pythonil{if age <= 3}.%
}\only<-10>{%
\item<7-> Wenn das \pythonilIdx{True} ergibt, dann setzen wir \pythonil{phase = \"infancy\"}, was bedeutet, dass sich die Person in der frühesten Kindheitsphase befindet.%
}\only<-12>{%
\item<8-> In diesem Fall ended auch der ganze \pythonil{if...elif..else} block und keine weiteren Bedingungen werden geprüft.%
}\only<-13>{%
\item<9-> Andernfalls, also wenn \pythonil{age <= 3} \pythonil{False} ergibt, also wenn \pythonil{age > 3} gilt, wird die nächste Bedingung geprüft.
}\only<-14>{%
\item<10-> Anstelle ein \pythonil{else} mit einem weiteren eingerückten \pythonil{if} können wir einfach \pythonil{elif age <= 6} ohne weitere Einrückung schreiben.%
}\only<-15>{%
\item<11-> Das \pythonilIdx{elif} funktioniert wie ein \pythonilIdx{else} gefolgt von einem eingrückten~\pythonil{if}.%
}\only<-16>{%
\item<12-> In unserem Fall prüft das \pythonil{elif} ob \pythonil{age <= 6} gilt~(nachdem wir an dieser Stelle schon wissen, dass \pythonil{age > 3}).%
}\only<-17>{%
\item<13-> Wenn ja, dann nennen wir diese Lebensphase~\pythonil{\"early childhood\"}.%
}\only<-18>{%
\item<14-> Wenn das zutrifft, dann ended der ganze \pythonil{if...elif...else}-Block und keine weiteren Bedigungen werden geprüft.%
}\only<-19>{%
\item<15-> Wenn nicht, also wenn \pythonil{age <= 6} \pythonilIdx{False} ergibt, dann geht es mit dem nächsten \pythonilIdx{elif} weiter.%
}\only<-20>{%
\item<16-> Nur dann wird also \pythonil{elif age <= 8} ausgefürt.%
}\only<-21>{%
\item<17-> Wenn diese Bedingung wahr ist, dann befindet sich die Person in der \pythonil{\"middle childhood\"} und sind mit dem \pythonil{if...elif...else} fertig.%
}%
\item<18-> Wenn \pythonil{age <= 8} nicht gilt, dann geht es wieder mit dem nächsten \pythonilIdx{elif} weiter.%
%
\item<19-> Und so weiter.%
%
\item<20-> Wenn selbst \pythonil{age < 80}, die Bedingung des letzten \pythonilIdx{elif}, auch \pythonilIdx{False} ist, dann wird der \pythonilIdx{else}-Block ausgeführt.%
%
\item<21-> Dann setzen wir \pythonil{phase = \"late adulthood\"}.%
%
\item<22-> Am Ende drucken wir die Lebensphase mit Hilfe von mit einem \glslink{fstring}{f-String} aus.%
\end{itemize}%
}}%
%
\gitLoadAndExecPython{conditionals:if_elif}{}{conditionals}{if_elif_example.py}{}%
\listingPythonAndOutput{}{conditionals:if_elif}{}{0.45}{0.17}{0.54}{0.92}%
\end{frame}%
%
\begin{frame}[t]%
\frametitle{Beispiel~(2)}%
%
\parbox{0.4\paperwidth}{\small{%%
\begin{itemize}%
\item Schauen wir uns nochmal unser Programm \programUrl{conditionals:if_else_nested} an.%
\only<-11>{%
\item<2-> Sieht eigentlich gut aus.%
\item<3-> Was sagt eigentlich \ruff\ dazu?%
\item<5-> Hm. Da hat \ruff\ eigentlich recht.%
\item<6-> Machen wir das mal.%
\item<8-> Das sieht wirklich besser aus.%
}%
\item<9-> Sehen Sie: Manchmal können wir was von einem \glslink{linter}{Linter} lernen.%
\item<10-> Diese Werkzeuge können uns schon echt helfen, die Kode-Qualität zu vergessen.%
\item<11-> Wir sparen eine Zeile Kode.%
\end{itemize}%
}}%
%
\gitExec{exec:conditionals:if_else_nested:ruff}{\programmingWithPythonCodeRepo}{.}{_scripts_/ruff.sh conditionals if_else_nested.py}%
\gitLoadAndExecPython{conditionals:if_elif_nested}{}{conditionals}{if_elif_nested.py}{}%
%
\listingPythonAndOutput{-3,7}{conditionals:if_else_nested}{}{0.45}{0.1}{0.54}{0.92}%
\listingOutput{4-6}{conditionals:if_else_nested:ruff}{style=tool_style}{0.45}{0.2}{0.54}{0.9}%
\listingPythonAndOutput{8-11}{conditionals:if_elif_nested}{}{0.45}{0.1}{0.54}{0.92}%
%
\uncover<12->{%
\bestPractice{elifOverElseIf}{Bevorzugen Sie \pythonilIdx{elif} über geschachtelte \pythonil{else ... if} Konstrukte\cite{M2022RAEFPLACFWIR}.}%
}%
%
\end{frame}%
%
\section{Inline bzw.\ Ternäres \texttt{if-else}}%
%
\begin{frame}[t]%
\frametitle{Use Case}%
%
\parbox{0.4\paperwidth}{\small{%%
\begin{itemize}%
\only<-5>{%
\item Ein häufiger Use-Case für \pythonil{if...else} ist die konditionelle Zuweisung von Werten auf Variablen.%
}%
%
\only<-6>{%
\item<2-> Hier haben wir so ein Beispiel.%
}%
%
\only<-7>{%
\item<3-> Wir wollen ein Programm schreiben, dass uns mitteilt, ob eine Zahl groß oder klein bzw.\ positiv oder negativ ist.%
}%
%
\only<-8>{%
\item<4-> Nehmen wir an, dass die Zahl in der Ganzzahl-Variablen \pythonil{number} gespeichert ist.%
}%
%
\only<-9>{%
\item<5-> Als Beispiel nehmen wir den Wert~\pythonil{100}.%
}%
%
\only<-10>{%
\item<6-> Im Beispiel nehmen wir auch an, dass eine Zahl \pythonil{number} deren Betrag $|\pythonil{number}|$ weniger als Zehn ist, als klein angesehen wird.%
}%
%
\only<-10>{%
\item<7-> Den Betrag einer Zahl können wir in \python\ mit der Funktion \pythonilIdx{abs} berechnen.%
}%
%
\only<-11>{%
\item<8-> Wir können also eine Alternative mit der Bedingung \pythonil{if abs(number) < 10:} bauen.%
}%
%
\only<-12>{%
\item<9-> Wenn die Bedigung zurtifft, dann speichern wir den String \pythonil{\"small\"} in der Variable \pythonil{size}.%
}%
%
\only<-13>{%
\item<10-> Trifft die Bedingung nicht zu, dann speichern wir stattdessen \pythonil{\"large\"} in \pythonil{size}.%
}%
\only<-14>{%%
\item<11-> Mit einer ähnlichen Alternative belegen wir die Variable \pythonil{sign}.%
}%
\item<12-> Wir speichern \pythonil{\"negative\"} in \pythonil{sign} wenn \pythonil{number < 0} zutrifft, \pythonil{\"positive\"} wenn \pythonil{number > 0}, und andernfalls, also wenn \pythonil{number == 0}, speichern wir \pythonil{\"unsigned\"}.%
%
\item<13-> Am Ende drucken wir das Ergebnis wieder unter Verwendung eines \glslink{fstring}{f-Strings}.%
%
\item<14-> Fragen wir mal \ruff, was es von unserem Kode hält.%
%
\item<16-> Hm. Interessant.%
\end{itemize}%
}}%
%
\gitLoadAndExecPython{conditionals:if_else_could_be_inline}{}{conditionals}{if_else_could_be_inline.py}{}%
\gitExec{exec:conditionals:if_else_could_be_inline:ruff}{\programmingWithPythonCodeRepo}{.}{_scripts_/ruff.sh conditionals if_else_could_be_inline.py}%
%
\listingPythonAndOutput{2-14}{conditionals:if_else_could_be_inline}{}{0.45}{0.1}{0.54}{0.92}%
\listingOutput{15-}{conditionals:if_else_could_be_inline:ruff}{style=tool_style}{0.45}{0.14}{0.54}{0.9}%
%
\end{frame}%
%
\begin{frame}[t]%
\frametitle{Inline bzw.\ Ternäres \texttt{if-else}}%
%
\begin{itemize}%
%
\only<-4>{%
\item In vielen Fällen brauchen wir \pythonil{if...else} bzw.\ \pythonil{if...elif...else} nur dazu, um zu entscheiden, welcher Wert in einer Variable gespeichert werden soll.%
%
\item<2-> In Programmiersprachen wie \pgls{C} und \pgls{Java} haben wir einen ternären Operator wie \textil{bedingung ? WertWennJa : WertWennFalsch} für solche Fäll.%
%
\item<3-> \python\ bietet uns auch eine kompakte Syntax -- \ruff\ hat uns gerade darauf hingewiesen.%
}%
%
\only<-6>{%
\item<4-> Das inline / ternäre \pythonil{if...else} Statement sieht wie folgt aus\cite{PEP308}:%
}%
%
\only<-7>{%
\item<6-> In der ersten Variante wird \pythonil{valueA} der Variable \pythonil{variable} zugewiesen when Bedingung \pythonil{conditionForUsingValueA} \pythonil{True} ergibt.%
}%
%
\only<-8>{%
\item<7-> Sonst wird \pythonil{valueB} zugwiesen.%
}%
%
\only<-9>{%
\item<8-> Solche Statements können beliebig geschachtelt werden, wie die zweite Variante zeigt.%
}%
%
\only<-10>{%
\item<9-> Hier wird wieder \pythonil{valueA} in \pythonil{variable} gespeichert wenn \pythonil{conditionForUsingValueA} wahr ist.%
}%
%
\only<-10>{%
\item<10-> Andernfalls wird \pythonil{valueB} benutzt, wenn \pythonil{conditionForUsingValueB} wahr ist und wenn nicht dann eben \pythonil{valueC}.%
}%
%
\item<11-> Das ternäre inline \pythonil{if...else} ist sehr nützlich.%
%
\end{itemize}%
%
\gitLoadPython{syntax/if_else_inline}{}{syntax/if_else_inline.py}{}%
\listingPythonSyntax{5-10}{syntax/if_else_inline}{0.1}{0.375}{0.8}{0.6}%%
%
\uncover<11->{%
\bestPractice{inlineIfThenElse}{Wenn ein \pythonil{if...else}-Statement nur dazu verwendet wird, zu entscheiden welcher Wert in einer Variable gespeichert wird, dann nehmen Sie stattdessen die ternäre inline Variante, da diese kompakter ist\cite{M2022RAEFPLACFWIR}.}%
}%
\end{frame}%
%
\begin{frame}[t]%
\frametitle{Beispiel}%
%
\parbox{0.4\paperwidth}{\small{%%
\begin{itemize}%
\item Schauen wir uns nochmal das vorige Beispiel an.%
\item<2-> Bauen wir es jetzt mit der ternären inline Variante vom \pythonil{if...else} nach.%
\item<4-> Wow. Wir brauchen jetzt nur noch 13 anstatt von 22 Zeilen Kode.%
\item<5-> Wieder ein Beispiel dafür, wie nützlich \glslink{linter}{Linter} seien können.%
\end{itemize}%
}}%
%
\gitLoadAndExecPython{conditionals:inline_if_else}{}{conditionals}{inline_if_else.py}{}%
\listingPythonAndOutput{-2}{conditionals:if_else_could_be_inline}{}{0.45}{0.1}{0.54}{0.92}%
\listingPythonAndOutput{3-}{conditionals:inline_if_else}{}{0.45}{0.1}{0.54}{0.92}%
%
\end{frame}%
%
\section{Zusammenfassung}%
%
\begin{frame}%
\frametitle{Zusammenfassung}%
\begin{itemize}%
\item Wir sind nun in der Lage, Programme zu bauen, die Entscheidungen basierend auf Daten treffen können.%
%
\item<2-> Vorher konnten wir nur einfache Berechnungen durchführen und die Ergebnisse einfacher Funktionen berechnen.%
%
\item<3-> Nun können unsere Variablen das Ergebnis einer Funktion~${\mathcal{A}}$ empfangen, wenn unsere Eingabedaten die Bedingung~$B$ erfüllen und andernfalls das Ergebnis einer Funktion~${\mathcal{C}}$.%
%
\item<4-> Das ist schon ziemlich cool.%
%
\item<5-> Wir könnten jetzt schon hard-kodierte Entscheidungsbäume implementieren~\cite{RN2022AIAMA,SSBD2014UMLFTTA} und im Grunde war ja \programUrl{conditionals:if_elif} ein Beispiel genau dafür.%
%
\item<6-> Mit Alternativen können wir im Grunde \inQuotes{über Instruktionen springen}.%
%
\item<6-> Mit der Ausnahme, dass wir Befehle jetzt auch übergehen können, werden unsere Programm allerdings immer noch in der Reihenfolge abgearbeitet, in der wir sie aufgeschrieben haben.%
%
\item<7-> Wir können Instruktionen noch nicht mehrfach ausführen bzw.\ im Kode \inQuotes{zurück springen}.%
%
\item<8-> Warten Sie mal auf die nächsten Einheiten\dots%
\end{itemize}%
%
\end{frame}%
%
\endPresentation%
\end{document}%%
\endinput%
%

\pdfminorversion=7%
\documentclass[aspectratio=169,mathserif,notheorems]{beamer}%
%
\xdef\bookbaseDir{../../bookbase}%
\xdef\sharedDir{../../shared}%
\RequirePackage{\bookbaseDir/styles/slides}%
\RequirePackage{\sharedDir/styles/styles}%
\toggleToGerman%
%
\subtitle{9.~Zwischenspiel:~Python Dokumentation und Informationsquellen}%
%
\begin{document}%
%
\startPresentation%
%
\section{Einleitung}%
%%
\begin{frame}%
\frametitle{Einleitung}%
\parbox{0.65\linewidth}{%
\begin{itemize}%
\item Ich habe Programmieren in den 1990ern mit \glsreset{turboPascal}\pgls{turboPascal}\cite{B1992TPV7UB} gelernt.%
\item<2-> Hauptsächlich habe ich es mir selber beigebracht, wobei ich zwei Quellen verwendet habe\only<-2>{.}\uncover<3->{:\begin{enumerate}%
\item Bücher%
\item<4-> Das großartige Hilfesystem von \pgls{turboPascal}%
\end{enumerate}}%
%
\item<5-> Dieses Hilfesystem hatte jeden einzelnen Befehl der Sprache aufgelistet, mit Beispielen!%
\item<6-> So konnte ich selbstständig neue Befehle lernen und ausprobieren.%
\item<7-> \emph{Wie können Sie neue Befehle für \python\ lernen und ausprobieren?}
\end{itemize}}%
%
\locateGraphic[Die Titelseite vom \citetitle{B1992TPV7UB}\cite{B1992TPV7UB}, copyright Borland.]{}{width=0.305\paperwidth}{graphics/turboPascal}{0.67}{0.1}%
\end{frame}%
%
\section{Die offizielle Dokumentation verwenden}%
%
\begin{frame}[t]%
\frametitle{Die offizielle Dokumentation verwenden}%
\begin{itemize}%
\only<-5,12->{%
\item Angenommen, Sie kennen den Namen eines Kommandos und wollen mehr darüber erfahren, z.B.~die genaue Bedeutung, Beschränkungen für die Werte der Parameter, mögliche Fehler, usw.%
\item<2-> Dafür gibt es einen kompletten Katalog mit Hilfe für \python:~die \citetitle{PSF:P3D}\cite{PSF:P3D} bei~\citeurl{PSF:P3D}.%
\item<3-> Dies sollte Ihr erster Ansatzpunkt sein, wenn Sie mehr über \python\ lernen wollen.%
\item<4-> Diese Hilfe steht auch in Chinesisch bereit!%
}%
\only<-6>{\item<5-> Nehmen wir an, Sie wollen mehr über die Funktion~\pythonil{ceil} erfahren.}%
\only<-6>{\item<6-> Wir besuchen die Website \citetitle{PSF:P3D} bei~\citeurl{PSF:P3D}.}%
\only<-7>{\item<7-> Wir können verschiedene Sprachen auswählen, auch Chinesisch {\dots} aber lassen wir es bei Englisch.}%
\only<-8>{\item<8-> Wir gegen \pythonil{ceil} in die Suchbox ein und klicken auf~\menu{Go}.}%
\only<-9>{\item<9-> Wir müssen nun etwas warten.}%
\only<-10>{\item<10-> Die Ergebnisse erscheinen. Wir klicken auf \pythonil{math.ceil}.}%
\only<-11>{\item<11-> Nun werden wir zur Bescheibung der Funktion gebracht.}%
\item<12-> Kennen wir den Namen der Funktion, die wir suchen, so finden wir bei der offiziellen \python\ Dokumentation die Beschreibung.%
\end{itemize}%
%
\locateGraphicTB{6}{width=0.55\paperwidth}{graphics/pythondoc/pythondoc01website}{0.225}{0.3}%
\locateGraphicTB{7}{width=0.55\paperwidth}{graphics/pythondoc/pythondoc02language}{0.225}{0.3}%
\locateGraphicTB{8}{width=0.55\paperwidth}{graphics/pythondoc/pythondoc03searchCeil}{0.225}{0.3}%
\locateGraphicTB{9}{width=0.55\paperwidth}{graphics/pythondoc/pythondoc04wait}{0.225}{0.3}%
\locateGraphicTB{10}{width=0.55\paperwidth}{graphics/pythondoc/pythondoc05results}{0.225}{0.3}%
\locateGraphicTB{11}{width=0.55\paperwidth}{graphics/pythondoc/pythondoc06mathceil}{0.225}{0.3}%
\end{frame}%
%
\section{Suchen mit einer Suchmaschine}%
%
\begin{frame}[t]%
\frametitle{Suchen mit einer Suchmaschine}%
\begin{itemize}%
\only<-4>{%
\item Kennen wir den Namen der Funktion, die wir suchen, so finden wir bei der offiziellen \python\ Dokumentation die Beschreibung.%
\item<2-> Was aber, wenn wir den Namen \alert{nicht} kennen?%
\item<3-> Nehmen wir an, wir suchen eine Funktion, die Zahlen aufrundet.%
}%
%
\only<-5>{\item<4-> Versuchen wir es zuerst wieder mit der offiziellen Dokumentation.}%
%
\only<-8>{\item<6-> Wir überlegen uns, wie man \inQuotes{aufrunden} auf Englisch formulieren kann.%
\uncover<7->{ Probieren wir es mit \inQuotes{round towards positive infinity}.%
\uncover<8->{ Wir geben das in die Suchmaske ein und klicken~\menu{Go}.}}%
}%
%
\only<-10>{\item<9-> Diesmal finden wir nichts nützliches.%
\uncover<10->{ Wir könnten zwar auf \inQuotes{\pythonil{math} -- Mathematical functions} klicken und uns durch den ganzen Katalog mathematischer Funktionen durcharbeiten {\dots} aber das ist vielleicht zu umständlich.}%
}%
\only<-12>{\item<11-> Öffnen wir stattdessen eine Suchmaschen (hier:~\url{https://bing.com}) und geben die selbe Frage dort ein.%
\uncover<12->{ Gleich der erste Link sieht interessant aus.}%
}%
%
\only<-13>{\item<13-> Er bringt uns zu einer Webseite über das Runden in \python: \citetitle{D2021RPT:A2024HTRNIP}\cite{D2021RPT:A2024HTRNIP} unter \citeurl{D2021RPT:A2024HTRNIP}.}%
\only<-15>{\item<14-> Wenn wir etwas herunterscrollen finden wir die gesuchte Information.%
\uncover<15->{ Sogar einen Link zur offiziellen Dokumentation.}%
}%
%
\only<16->{%
\item<16-> Wir folgen diesem Link. Wir schlagen \alert{immer} alle gefunden Informationen nochmal in der offiziellen Dokumentation nach.%
%
\item<17-> Wir können also durchaus mit einer Suchmaschine auch Funktionen finden, deren Namen wir nicht kennen.%
}%
%
\end{itemize}%
%
\locateGraphicTB{5}{width=0.55\paperwidth}{graphics/pythondoc/pythondoc01website}{0.225}{0.3}%
\locateGraphicTB{6-8}{width=0.55\paperwidth}{graphics/rounding/rounding01searchInDocumentation}{0.225}{0.3}%
\locateGraphicTB{9-10}{width=0.55\paperwidth}{graphics/rounding/rounding02foundNothingUseful}{0.225}{0.3}%
\locateGraphicTB{11-12}{width=0.55\paperwidth}{graphics/rounding/rounding03searchEngine}{0.225}{0.3}%
\locateGraphicTB{13}{width=0.55\paperwidth}{graphics/rounding/rounding04realPython}{0.225}{0.3}%
\locateGraphicTB{14-15}{width=0.55\paperwidth}{graphics/rounding/rounding05realPythonScrolledToCeil}{0.225}{0.3}%
\locateGraphicTB{16}{width=0.55\paperwidth}{graphics/rounding/rounding06foundCeilInPythonDoc}{0.225}{0.3}%
%
\end{frame}%
%
\section{Suchen mit Community-Portalen}%
%
\begin{frame}[t]%
\frametitle{Suchen mit Community-Portalen}%
\begin{itemize}%
\only<-3>{%
\item Es gibt sehr viele Community-Portale rund ums Programmieren.%
\item<2-> Auch in diesen Portalen können wir nach nützlichen Informationen suchen.%
}%
\only<-4>{\item<3-> Probieren wir das mal aus und besuchen das bekannte Portal \citetitle{SE:SO}\cite{SE:SO} bei \citeurl{SE:SO}.}
%
\only<-5>{\item<5-> Warnung: Manchmal fordert uns diese Webseite auf, weitere \pglspl{javascript} von anderen Quellen zu laden. Wenn das passiert dann machen wir das.}%
%
\only<-6>{\item<6-> Wenn die Webseite voll geladen ist, können wir unsere Frage eingeben.}%
%
\only<-7>{\item<7-> Wir probieren es diesmal mit \inQuotes{python round towards infinity}.}%
%
\only<-9>{\item<8-> Alle Fragen, die der Community gestellt wurden, und die so ähnlich klingen, werden gelistet.%
\uncover<9->{ Wir müssen etwas herunterscrollen, um eine Frage zu finden, die passend klingt.}%
}%
%
\only<-10>{\item<10-> Wir finden die Frage \citetitle{SE:SO:HTIDWRTIIP}\cite{SE:SO:HTIDWRTIIP}, welche passend klingt, und klicken darauf.}%
%
\only<-12>{\item<11-> Die Frage klingt tatsächlich passend, ist aber eigenartig.%
\uncover<12->{ Es wird z.B. geschrieben, dass \pythonil{3/2} eins ergibt, was unserem Verständnis des Divisionsoperators widerspricht.}%
}%
%
\only<-14>{\item<13-> Weiter unten stellen wir fest, dass die Frage für \python~2 gestellt war, nicht für \python~3. Unter \python~2 funktionierte der Operator \pythonil{/} wir der \pythonil{//}\nobreakdashes-Operator unter \python~3\cite{PEP238}.%
\uncover<14->{ Wir finden aber eine gute Erklärung und einen Verweis auf \pythonil{ceil}, was unser Problem löst.}%
}%
%
\item<15-> Auch mit Community-Portalen können wir also gute Antworten auf unsere Fragen finden.%
%
\end{itemize}%
%
\locateGraphicTB{4-5}{width=0.55\paperwidth}{graphics/stackOverflow/stackOverflow01websiteLoading}{0.225}{0.3}%
\locateGraphicTB{6}{width=0.55\paperwidth}{graphics/stackOverflow/stackOverflow02websiteLoadedEnterQuery}{0.225}{0.3}%
\locateGraphicTB{7}{width=0.55\paperwidth}{graphics/stackOverflow/stackOverflow03queryEntered}{0.225}{0.3}%
\locateGraphicTB{8-9}{width=0.55\paperwidth}{graphics/stackOverflow/stackOverflow04queryAnsweres}{0.225}{0.3}%
\locateGraphicTB{10}{width=0.55\paperwidth}{graphics/stackOverflow/stackOverflow05foundVaguelyRelatedQuestion}{0.225}{0.3}%
\locateGraphicTB{11-12}{width=0.55\paperwidth}{graphics/stackOverflow/stackOverflow06questionFits}{0.225}{0.3}%
\locateGraphicTB{13}{width=0.55\paperwidth}{graphics/stackOverflow/stackOverflow07solutionFits}{0.225}{0.3}%
\end{frame}%
%
\begin{frame}%
\frametitle{Zwischenstand}%
\begin{itemize}%
\item Wir können nun Antworten auf unsere Fragen auf drei Wegen finden\only<-1>{.}\uncover<2->{:%
\begin{enumerate}%
\item über die offizielle Dokumentation\uncover<3->{,}%
\item<3-> über eine Suchmaschine, die uns zu anderen Webseiten im Internet bringt\uncover<4->{, und}%
\item<4-> über Community-Portale.%
\end{enumerate}%
}%
\item<5-> \alert{Nur die offizielle Dokumentation unserer Programmiersprache ist autoritativ.}%
\item<6-> Nur der offiziellen Dokumentation können wir voll vertrauen.%
\item<7-> Alle anderen Quellen sind von irgendwelchen anderen Leuten geschrieben.%
\item<8-> Sie können falsch sein.\uncover<9->{ Oder veraltet.}%
\item<10-> Alles, was wir inderen Quellen finden, muss also in der offiziellen Dokumentation nachgeprüft werden.%
\end{itemize}%
\end{frame}%
%
\section{Suchen mit AI}%
%
\begin{frame}[t]%
\frametitle{Suchen mit AI: Ausprobieren}%
\begin{itemize}%
\only<-3>{%
\item Können wir auch mit einer \glsFull{AI} nach Antworten suchen?%
\item<2-> Ja klar, warum nicht.%
\item<3-> Probieren wir das mal aus.%
}%
%
\only<-4>{\item<4-> Wir öffnen die BaiDu~(百度) Suchseite unter~\url{https://baidu.com} und klicken auf \inQuotes{AI搜索已接入DeepSeek\nobreakdashes-V3最新版模型}.%
}%
%
\only<-5>{\item<5-> Wir kommen auf die Seite \inQuotes{Hi,AI搜索已支持R1满血版,快来试试吧!}, auf der wir unseren Prompt f+r das DeepSeek\nobreakdashes-R1 Modell\cite{DAGYZSZXZMWBZYWWGSLGLXWWFLZDZRDCJLLDLHCLZBXWDXGQLGLWCYQLCNLCDHGGHYWZZWZXXZZTLWLTHZWCDGZPWCJCLZCYWYZPL2024DRIRCILVRL} in das Eingabefeld unten eintippen können%
}%
%
\only<-6>{\item<6-> Wir schreiben \inQuotes{How can I round floating point numbers towards positive infinity in Python?} und drücken~\keys{\enter}.%%
}%
%
\only<-7>{\item<7-> Die \pgls{AI} holt zu einer langatmigen Antwort aus.}%
\only<-8>{\item<8-> Diese Antwort beinhaltet auch die richtige Lösung.}%
\only<-9>{\item<9-> Sogar mit einem Beispiel!}%
\only<-10>{\item<10-> Die vollständige Antwort ist lang und umständlich.}%
\only<-11>{\item<11-> Sie beinhaltet aber auch die \textcolor{blue}{richtige Antwort}, die \pythonil{ceil}~Funktion.}%
\only<-12>{\item<12-> Aber eben auch Fehler: \textcolor{red}{Like 2.0000000001, would ceil round it to 3? Probably not}.}%%
%
\end{itemize}%
%
\locateGraphicTB{4}{width=0.55\paperwidth}{graphics/aiBaidu/ai01baidu}{0.225}{0.3}%
\locateGraphicTB{5}{width=0.55\paperwidth}{graphics/aiBaidu/ai02deepseek}{0.225}{0.3}%
\locateGraphicTB{6}{width=0.55\paperwidth}{graphics/aiBaidu/ai03deepseekQuery}{0.225}{0.3}%
\locateGraphicTB{7}{width=0.55\paperwidth}{graphics/aiBaidu/ai04deepseekAnswer1}{0.225}{0.3}%
\locateGraphicTB{8}{width=0.55\paperwidth}{graphics/aiBaidu/ai05deepseekAnswer2}{0.225}{0.3}%
\locateGraphicTB{9}{width=0.55\paperwidth}{graphics/aiBaidu/ai06deepseekAnswer3}{0.225}{0.3}%
%
\locate{10-}{
\parbox{0.98\paperwidth}{\noindent%
\centering%
\emph{\footnotesize{Teil der Ausgabe der DeepSeek-Suche von Baidu vom 2025\nobreakdashes-04\nobreakdashes-27, manuell hinzugefügte Formatierung.}}\\%
\resizebox{0.98\paperwidth}{!}{%
\fcolorbox{yellow!20!black}{yellow!10!white}{%
\resizebox{0.98\paperwidth}{!}{%
\parbox{1.4\paperwidth}{\noindent%
{\tiny{Okay, I need to figure out how to round floating point numbers towards positive infinity in Python.
Let's see, rounding towards positive infinity is also known as ceiling.
{\color{blue}{So, maybe there's a function called \texttt{ceil} in Python?}}\smallskip

Wait, I remember that Python has a \texttt{math} module.
Let me check.
{\color{blue}{Yes, the \texttt{math} module has a \texttt{ceil} function.}}
So, \texttt{math.ceil()} should do the trick.
But wait, how does it handle different numbers?
For example, if the number is already an integer, does it leave it as is?
And for negative numbers, does it round them up towards zero or away?\smallskip

Let me think.
The ceiling function in mathematics rounds a number up to the nearest integer.
So, for positive numbers, even if they are already integers, it stays the same.
For negative numbers, it would round towards zero.
For example, \texttt{ceil(-2.3)} would be \texttt{-2}, which is greater than \texttt{-2.3}.
So that's correct for positive infinity direction.\smallskip

{\color{red}{But how about when dealing with floating points that are very close to an integer due to precision issues?
Like \texttt{2.0000000001}, would \texttt{ceil} round it to \texttt{3}?
Probably not, because the function should handle that as per the actual value.}}
Wait, but floating points can have precision problems.
For instance, \texttt{1.1} might not be represented exactly.
So \texttt{ceil(1.1)} should give \texttt{2}, but if due to some calculation the number is actually stored as slightly less than \texttt{1.1}, would that affect the result?
Hmm, maybe that's a potential issue, but the question is about rounding towards positive infinity, so assuming the input is correct, \texttt{math.ceil} should work.\smallskip

Wait, what about numbers that are exactly halfway between two integers?
Like \texttt{2.5}.
Does \texttt{ceil(2.5)} round to \texttt{3}? Yes, because ceil rounds up regardless of the fractional part.
So, even \texttt{2.1} would go to \texttt{3}, right?
No, wait: \texttt{ceil(2.1)} is \texttt{3}?
No, wait no.
Wait, \texttt{ceil(2.1)} is \texttt{3}?
Wait no!
Wait, no.
Wait, no, that's not right.
Wait, the ceiling function rounds up to the smallest integer greater than or equal to the number.
So, \texttt{2.1}'s ceiling is \texttt{3}?
No, wait, \texttt{2.1} is between \texttt{2} and \texttt{3}.
The smallest integer greater than or equal to \texttt{2.1} is \texttt{3}?
Wait, no.
Wait \texttt{2.1} is greater than \texttt{2}, so the smallest integer greater or equal is \texttt{3}?
Wait, no, wait, \texttt{2.1}'s ceiling is \texttt{3}?
No, no.
Wait, \texttt{2.1} is between \texttt{2} and \texttt{3}.
So the ceiling is \texttt{3}.
Wait, but \texttt{2.0}'s ceiling is \texttt{2.0}.
So yes, \texttt{ceil(2.1)} would be \texttt{3}?
Wait, wait, I'm getting confused.
Let me think again.\smallskip

Wait, the ceiling of \texttt{2.1} is \texttt{3}?
No, that's not right.
Wait, the ceiling function rounds up to the next integer.
So \texttt{2.1} would have a ceiling of \texttt{3}?
No, wait no.
Wait, no, the ceiling of \texttt{2.1} is \texttt{3}?
No, that's not correct.
Wait, \texttt{2.1} is \texttt{2.1}, so the smallest integer greater than or equal to \texttt{2.1} is \texttt{3}?
No! Wait, \texttt{2} is less than \texttt{2.1}, but \texttt{3} is greater than \texttt{2.1}.
Wait, no, \texttt{2.1} is between \texttt{2} and \texttt{3}.
The ceiling is the smallest integer that is greater than or equal.
So \texttt{3} is greater than \texttt{2.1}, but \texttt{2} is less than \texttt{2.1}.
So the ceiling would be \texttt{3}?
No, that's not right.
Wait, no, \texttt{2.1} is greater than \texttt{2}, so the ceiling is the smallest integer that is not less than the number.
So the ceiling is \texttt{3}?
Wait, no.
Wait, \texttt{2.1} is \texttt{2.1}, so the next integer up is \texttt{3}.
So yes, \texttt{ceil(2.1)} is \texttt{3}?
Wait, no.
Wait, I think I made a mistake here.
Because the ceiling of \texttt{2.1} is \texttt{3}?
No, the ceiling of \texttt{2.1} is actually \texttt{3}.
Wait, no, that's not correct.
Wait, the ceiling of \texttt{2.0} is \texttt{2.0}.
The ceiling of \texttt{2.0} is \texttt{2.0}.
For \texttt{2.1}, the next integer higher is \texttt{3.0}.
So \texttt{ceil(2.1)} is \texttt{3.0}.
Wait, but that's not right.
Wait, \texttt{2.1} is less than \texttt{3.0}, but the ceiling is the smallest integer greater than or equal.
So for example, \texttt{ceil(2.0)} is \texttt{2.0}, \texttt{ceil(2.1)} is \texttt{3.0}?
That can't be right.
Wait, maybe I'm mixing up ceiling and floor functions.
Wait, the floor function rounds down.
So \texttt{floor(2.1)} is \texttt{2}.
The ceiling function rounds up.
So \texttt{ceil(2.1)} is \texttt{3.0}?
That would be the case.
But that's a big jump, but yes.
For example, in Python:%
%
\begin{center}%
\parbox{0.7\linewidth}{%
\texttt{import math}\\%
\texttt{math.ceil(2.1)}\\%
\texttt{3}%
}\end{center}%
%
Is that true?
Let me check.
Wait, no, that's not correct.
Wait, I'm confusing something here.
Let me check an example.
Let me try to recall: \texttt{ceil(1.2)} is \texttt{2}?
No.
Wait, no.
Wait, \texttt{ceil(1.2)} is \texttt{2}.
Yes, because \texttt{2} is the smallest integer greater than or equal to \texttt{1.2}.
Wait, no.
Wait, \texttt{1.2} is between \texttt{1} and \texttt{2}.
So the smallest integer greater than or equal to \texttt{1.2} is \texttt{2}?
No, that's not correct.
Wait, \texttt{1.2} is greater than \texttt{1}, but \texttt{2} is larger than \texttt{1.2}, but the ``smallest integer greater than or equal to'' \texttt{1.2} is \texttt{2}?
No.
Wait, no.
Wait, the integers around \texttt{1.2} are \texttt{1} and \texttt{2}.
\texttt{1.2} is greater than \texttt{1}, so the smallest integer that is greater than or equal is \texttt{2}?
No.
Wait, no.
Wait, \texttt{1.2} is \texttt{1.2}, so the next integer higher is \texttt{2}.
So the ceiling is \texttt{2}.
Yes.
So \texttt{ceil(1.2)} is \texttt{2}.
So in that case, yes, that's correct.\smallskip

Wait, but then for numbers like \texttt{2.5}, \texttt{ceil(2.5)} is \texttt{3}.
So regardless of the decimal part, it rounds up.
So that's correct.\smallskip

But in Python, how does this work?
Let me think of code examples.
If I do:%
%
{\color{blue}{%
\begin{center}%
\parbox{0.7\linewidth}{%
\texttt{import math}\\%
\texttt{print(math.ceil(2.1)) \# should output 3?}%
}\end{center}}}%
%
Wait, but when I run this code, what's the actual output?
Wait, maybe I'm making a mistake here.
Let me check.\smallskip

Wait, actually, if I run \texttt{math.ceil(2.1)}, the result is \texttt{3}.
Let me confirm.\smallskip

Wait, no, wait, no.
Wait, in Python, \texttt{math.ceil(2.1)} returns \texttt{3.0}?
Let me think again.
Wait, no, \texttt{2.1}'s ceiling is \texttt{3.0}?
That can't be right.
Wait, maybe I'm thinking of the integer division or something else.
Wait, no. Wait, the ceiling function works as in mathematics.
So for any non-integer, the ceiling is the next integer higher.\smallskip

Wait, but \texttt{2.0}'s ceiling is \texttt{2.0}, \texttt{2.1}'s ceiling is \texttt{3.0}.
Wait, that's how the function works.
So, yes, \texttt{math.ceil(2.1)} would return \texttt{3.0}.\smallskip

But wait, in reality, when I test this in Python, maybe it's different.
Let me think.
Wait, perhaps I'm making a mistake here.
Wait, let's see:\smallskip

For example, \texttt{math.ceil(2.1)} returns \texttt{3}, but \texttt{math.ceil(2.0)} returns \texttt{2.0}.
So, yes.\smallskip

So, for any positive number with a fractional part, ceil will round it up to the next integer.
For negative numbers, \texttt{ceil} will round towards zero.
For example, \texttt{ceil(-1.2)} is \texttt{-1}, because \texttt{-1} is greater than \texttt{-1.2}, and is the smallest integer that satisfies that.\smallskip

So, \texttt{math.ceil} is the correct function to use here.
Therefore, the answer is to use \texttt{math.ceil} from the \texttt{math} module.\smallskip

But wait, the user asked how to round towards positive infinity.
So, for positive numbers, that's rounding up (\texttt{ceil}), but for negative numbers, rounding towards positive infinity would also require moving towards zero, which \texttt{ceil} does.
So, yes, \texttt{math.ceil()} is correct.\smallskip

But let me check another case.
For example, if the number is already an integer, like \texttt{5.0}, then \texttt{ceil(5.0)} is \texttt{5.0,} which is correct.
If it's a negative number with fractional part, like \texttt{-3.2}, \texttt{ceil(-3.2)} is \texttt{-3.0}, which is towards positive infinity (since \texttt{-3} is greater than \texttt{-3.2}).
So that's correct.\smallskip

So, the conclusion is:
to round a floating point number towards positive infinity in Python, use the \texttt{math.ceil()} function from the \texttt{math} module.
You need to import \texttt{math} and then call \texttt{math.ceil(number)}.
The function returns a \texttt{float} if the input is a \texttt{float}, but if you want an integer, you can cast it to an \texttt{int}.\smallskip

Wait, but \texttt{math.ceil(2.1)} returns \texttt{3.0}, which is a \texttt{float}.
So if you need an integer, you have to convert it.
For example, \texttt{int(math.ceil(2.1))} gives \texttt{3}.
But the question is about rounding the floating point number towards positive infinity.
So depending on whether the user wants the result as a \texttt{float} or integer, but the default behavior of \texttt{math.ceil} returns a \texttt{float} if the input is a \texttt{float}.
For example, \texttt{math.ceil(2)} (if \texttt{2} is an integer) would return \texttt{2}, but if you pass \texttt{2.0}, it returns \texttt{2.0}.\smallskip

So the user needs to be aware of that.
But the main answer is to use \texttt{math.ceil}.\smallskip

Wait, but what about numbers that are already integers?
For example, \texttt{3.0}, \texttt{math.ceil(3.0)} returns \texttt{3.0}.
So if the user wants to ensure it's an integer, they can cast it.
But maybe the user just wants to round towards positive infinity, regardless of data type.\smallskip

So, the answer is to import math and use \texttt{math.ceil()}.\smallskip

Another thing to consider:
what if the number is very large and cannot be represented as an integer?
For example, \texttt{1e300} is a float, and \texttt{math.ceil(1e300)} would return the same \texttt{float}, which is the integer value.
But since Python can handle arbitrarily large integers, converting it to an integer would be possible.\smallskip

So, in summary, the way to round a floating point number towards positive infinity in Python is to use the \texttt{math.ceil} function.
Make sure to import the math module first.\smallskip

To round a floating point number towards positive infinity (ceiling) in Python, you can use the \texttt{math.ceil()} function from the math module.
This function returns the smallest integer greater than or equal to the given number, which effectively rounds the number up.
Here's how to do it:%
%
\begin{enumerate}%
\item Import the \texttt{math} module.%
\item Use \texttt{math.ceil()} with the number you want to round.%
\end{enumerate}%
%
Example Code:%
\begin{center}%
\parbox{0.7\linewidth}{%
\texttt{import math}\\%
\strut\\%
\texttt{number = 2.3}\\%
\texttt{rounded = math.ceil(number)}\\%
\texttt{print(rounded)  \# Output: 3}\\%
\strut\\%
\texttt{negative\_number = -2.3}\\%
\texttt{rounded\_negative = math.ceil(negative\_number)}\\%
\texttt{print(rounded\_negative)  \# Output: -2}\\%
}\end{center}%
%
Notes:

\texttt{math.ceil()} returns a \texttt{float} if the input is a \texttt{float} (e.g., \texttt{3.0} for \texttt{math.ceil(2.1)}).
If you need an integer, cast the result to \texttt{int}: \texttt{int(math.ceil(number))}.
For positive numbers, it rounds up to the next integer.
For negative numbers, it rounds towards zero (the direction of positive infinity).

This method correctly handles all cases, including integers (returns them unchanged) and very large numbers.
\dots%
}}%
}}}}}%
}{0.01}{0.23}%
%
\end{frame}%
%
%
\begin{frame}%
\frametitle{Suchen mit AI: Nachteile (1)}%
\begin{itemize}%
%
\item Antworten, die wir von Webseiten, \pgls{AI}-Werkzeugen, oder anderen nicht-autoritativen Quellen bekommen, können falsch, unvollständig, oder veraltet sein.%
%
\item<2-> Wir dürfen nur Kode verwenden, den wir vollständig verstehen.\uncover<3->{ Wir dürfen niemals Funktionen vertrauen, deren offizelle Dokumentation wir nicht gelesen haben.}%
%
\item<4-> Kode, den wir nicht mit unseren eigenen Worten erklären können, ist \alert{falsch}.%
%
\item<5-> \pglspl{AI} kann uns gute Vorschläge für \inQuotes{Allerweltsfragen} geben, für Probleme, die oftmals gestellt und gelöst wurden.%
\uncover<6->{ Sind Sie jedoch Masterstudent oder Doktorand und arbeiten wissenschaftlich, dann wurde Ihr Problem eben gerade nicht von vielen Leuten gestellt und gelöst.}%%
%
\item<7-> Werkzeuge wie \glsFullpl{LLM} laden zu Faulheit ein.%
%
\item<8-> Ihr Wert ist Ihr Können, Ihre Kenntnis, und Ihre Erfahrung.%
\uncover<9->{ Bestehen diese darin, Lösungen einer \pgls{AI} zu kopieren, dann kann man Sie auch durch einen billigeren Schulabgänger ersetzen.}%
%
\end{itemize}%
%
\end{frame}%
%
%
\begin{frame}[t]%
\frametitle{Suchen mit AI: Nachteile (2)}%
\begin{itemize}%
\only<-5,8->{%
%
\item Ein Wichtiges Features von Software sind durchgängige, einheitliche Architektur und Stiel.\uncover<2->{ Ein aus \pgls{AI}-Versatzstücken zusammengeschustertes System lässt sich nicht warten oder verstehen.}%
%
\item<3-> \pgls{AI}-Werkzeuge können Fehler machen, die kein Mensch machen würde.\uncover<4->{ %
Ein Beispiel findet sich in \citetitle{BSHETB:VSK2025CIC}\cite{BSHETB:VSK2025CIC}, wo dokumentiert wird, wie Microsoft Copilot eine Klasse auf eine verwirrende Art umbenannt hat, was zu einem besonders schwer zu findenden Fehler geführt hat.}%
}%
%
\item<5-> Hier ist ein Beispiel, wie eine Vibe-Coding \pgls{AI} die \glslink{db}{Produktivdatenbank} eines Unternehmens gelöscht hat.%
%
\item<8-> \pgls{AI}-Werkzeuge schlagen manchmal das Benutzen von \alert{nicht existierenden} Packages vor, was zu großen Sicherheitsproblemen im Kode führen kann\cite{AT:G2025AGCCBADFTSSCHW}.%
%
\end{itemize}%
%
\locateGraphicTB[Quelle \& Copyright:~\bracketCite{T2025ACPGRDCFADECDRCAAAESIMACEIJADAPD}]{6}{width=0.45\paperwidth}{graphics/aiDBdelete/aiDBdeleteTomsHW}{0.033333333}{0.3}%
\locateGraphicTB[Quelle \& Copyright:~\bracketCite{N2025AAPCTWOASCDTAFACFOMP}]{6}{width=0.45\paperwidth}{graphics/aiDBdelete/aiDBdeleteFortune}{0.516666666}{0.3}%
\locateGraphicTB[Quelle \& Copyright:~\bracketCite{S2025AMRCKLPUVR}]{7}{width=0.95\paperwidth}{graphics/aiDBdelete/aiDBdeleteGolem}{0.025}{0.3}%
\end{frame}%
%
\begin{frame}%
\frametitle{Suchen mit AI: Nachteile (3)}%
\begin{itemize}%
\item In der Grundschule haben Sie das Rechnen gelernt.%
\item<2-> Haben Sie von Anfang an einen Taschenrechner bekommen?%
\item<3-> Nein.%
\item<4-> Warum nicht?%
\item<5-> Weil Sie dann nicht gelernt hätten, zu Rechnen.%
\item<6-> Sie könnten zwar den Taschenrechner bedienen und Aufgaben lösen.%
\item<7-> Aber Sie hätten überhaupt gar nicht verstanden, wie Rechnen und Mathematik funktionieren.%
\item<8-> Der Sinn dieses Kurses ist es, das Programmieren zu lernen.%
\item<9-> Wenn Sie Lösungen durch eine \pgls{AI} generieren lassen, werden Sie das nie lernen.%
\end{itemize}%
\end{frame}%
%
\section{Weitere Elemente der Offiziellen Dokumentation}%
%
\begin{frame}[t]%
\frametitle{Weitere Elemente der Offiziellen Dokumentation}%
\begin{itemize}%
\only<-1>{%
\item Es gibt noch weitere Dokumente der offiziellen \python~Dokumentation.%
}%
\only<-2,6->{%
\item<2-> Die \citetitle{PSF:P3D:PSAU} Seite\cite{PSF:P3D:PSAU} unter~\citeurl{PSF:P3D:PSAU}.%
}%
\only<-3,6->{%
\item<3-> Die \citetitle{PSF:P3D:TPT} Seite\cite{PSF:P3D:TPT} unter~\citeurl{PSF:P3D:TPT}.%
}%
\only<-4,6->{%
\item<4-> Die \citetitle{PSF:P3D:TPSL} Seite\cite{PSF:P3D:TPSL} unter~\citeurl{PSF:P3D:TPSL}.%
}%
\item<5-> Die \citetitle{PEP0} Liste\cite{PEP0} unter~\citeurl{PEP0}.%
\item<6-> Alle davon sind Teil der offiziellen \python~Dokumentation.%
\item<7-> Bitte lernen Sie, diese Seiten zu lesen und zu benutzen.%
\end{itemize}%
%
\locateGraphicTB{2}{width=0.65\paperwidth}{graphics/pythondocOther/pythondocOther01setupAndUsage}{0.175}{0.25}%
\locateGraphicTB{3}{width=0.65\paperwidth}{graphics/pythondocOther/pythondocOther02tutorial}{0.175}{0.25}%
\locateGraphicTB{4}{width=0.65\paperwidth}{graphics/pythondocOther/pythondocOther03standardLibrary}{0.175}{0.25}%
\locateGraphicTB{5}{width=0.65\paperwidth}{graphics/pythondocOther/pythondocOther04pep0}{0.175}{0.25}%
\end{frame}%
%
\section{Zusammenfassung}%
%
\begin{frame}%
\frametitle{Zusammenfassung}%
\begin{itemize}%
\item Ein wichtiges Werkzeug für alle Programmierer ist die offizielle Dokumentation der verwendeten Programmiersprache oder Bibliotheken.%
\item<2-> Wir können (nur) der offiziellen Dokumentation vertrauen.%
\item<3-> Es ist eine wichtige Fähigkeit, in der Lage zu seien, sich hinzusetzen und eine Dokumentation zu lesen und zu verstehen.%
\item<4-> Wir können vielfältige andere Werkzeuge benutzen, um Informationen zu finden\only<-4>{.}\uncover<5->{:%
\begin{enumerate}%
\item Suchmaschinen, die uns zu anderen Webseiten führen\uncover<6->{,}%
\item<6-> Community-Portale, in denen wir Fragen durchsuchen und selber stellen können\uncover<7->{, und}%
\item<7-> \pgls{AI}-Werkzeuge, die uns Kode vorschlagen können.%
\end{enumerate}}%
%
\item<8-> Alle diese Werkzeuge können uns helfen.%
\item<9-> Wir dürfen aber nur Kode verwenden, den wir selbst vollständig verstehen.%
\item<10-> Wir müssen immer alle Funktionen, die wir nicht kennen, in der offiziellen autoritativen Dokumentation nachschlagen.%
\end{itemize}%
\end{frame}%
%
\endPresentation%
\end{document}%%
\endinput%
%

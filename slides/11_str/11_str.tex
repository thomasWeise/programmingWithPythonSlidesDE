\pdfminorversion=7%
\documentclass[aspectratio=169,mathserif,notheorems]{beamer}%
%
\xdef\bookbaseDir{../../bookbase}%
\xdef\sharedDir{../../shared}%
\RequirePackage{\bookbaseDir/styles/slides}%
\RequirePackage{\sharedDir/styles/styles}%
\toggleToGerman%
%
\subtitle{11.~Der Datentyp str}%
%
\begin{document}%
%
\startPresentation%
%
\section{Einleitung}%
%%
\begin{frame}%
\frametitle{Einleitung}%
\begin{itemize}%
\item Der vierte wichtige grundlegende Datentyp in \python\ sind Zeichenketten, Texte, auch genannt \emph{Strings}.\cite{PSF:P3D:TPSL:TSTS}%
\item<2-> Zeichenketten sind beliebig lange Sequenzen von Text-Zeichen.%
\item<3-> In \python\ sind sie durch den Datentyp \pythonil{str} repräsentiert.%
\item<4-> Wir haben sie bereits oftmals implizit oder explizit genutzt, z.B. in unserem ersten Programm, das einfach \pythonil{\"Hello World\"} ausgedruckt hat.%
\item<5-> \pythonil{\"Hello World\"} is so ein Text-String.%
\end{itemize}%
\end{frame}%
%
%
\section{Grundlegende Operationen}%
%
\begin{frame}[t]%
\frametitle{Strings Definieren, Verketten, und Indizieren}%
\begin{itemize}%
%
\only<-1,53->{\item Schauen wir uns die grundlegenden Operationen für Strings an.}%
%
\only<-2>{\item<2-> Wir öffnen ein Terminal (Unter \ubuntu\ \linux\ durch Drücken von \ubuntuTerminal, unter \microsoftWindows\ durch \windowsTerminal.)}%
%
\only<-3>{\item<3-> Wir schreiben \bashil{python3} und drücken~\keys{\enter}.}%
%
\only<-4>{\item<4-> Der \python-Interpreter startet.}%
%
\only<-9,12>{\item<5-> Es gibt zwei grundlegende Methoden, Strings zu definieren.\uncover<6->{\only<-7>{ %
Entweder mit einfachen oder doppelten Anführungszeichen.}\uncover<7->{ %
Probieren wir es zuerst mit den doppelten Anführungszeichen und schreiben \pythonil{\"Hello World!\"}.\uncover<8->{ %
Beachten Sie, dass die Anführungszeichen selbst nicht zum String gehören, sondern ihn nur begrenzen.\uncover<9->{ %
Der Text wird uns wieder ausgegeben.%
}}}}}%
%
\only<-12>{\item<10-> Nun probieren wir es mit einfachen Anführungszeichen und schreiben \pythonil{'Hello World!'}.\uncover<11->{ Und das wird uns auch wieder ausgegeben.%
}}%
%
\only<-14>{\item<13-> Strings können mit \pythonil{+} aneinander angehängt (verkettet) werden.\uncover<14->{ Sie ergeben dann ein einzigen String, der aus allen Teilstrings in der richtigen Reihenfolge besteht.%
}}%
%
\only<-16>{\item<15-> Die Funktion \pythonil{len(x)} liefert uns die Länge des Strings~\pythonil{x}, also die Anzahl der Zeichen in der Zeichenkette.\uncover<16->{ \inQuotes{Hello} besteht aus fünf Zeichen.%
}}%
%
\only<-18>{\item<17-> Wir können auch einzelne Zeichen aus einem String herausholen. \pythonil{x[i]} liefert das \mbox{$\pythonil{i}+1$\nobreakdashes-te} Zeichen.\uncover<18->{ Das erste Zeichen ist also an Index~0, bei \inQuotes{Hello} ist das~\inQuotes{H}. Dieses Zeichen ist natürlich wieder ein~\pythonil{str}~(nur mit Länge~1).%
}}%
%
\only<-20>{\item<19-> Das zweite Zeichen bekommen wir durch~\pythonil{[1]}.\uncover<20->{ Bei \inQuotes{Hello} ist das~\inQuotes{e}.%
}}%
%
\only<-22>{\item<21-> Das dritte Zeichen bekommen wir durch~\pythonil{[2]}.\uncover<22->{ Bei \inQuotes{Hello} ist das~\inQuotes{l}.%
}}%
%
\only<-24>{\item<23-> Das vierte Zeichen bekommen wir durch~\pythonil{[3]}.\uncover<24->{ Bei \inQuotes{Hello} ist das~\inQuotes{l}.%
}}%
%
\only<-26>{\item<25-> Das fünfte Zeichen bekommen wir durch~\pythonil{[4]}.\uncover<26->{ Bei \inQuotes{Hello} ist das~\inQuotes{o}.%
}}%
%
\only<-28>{\item<27-> \inQuotes{Hello} hat fünf Zeichen. Was passiert, wenn wir versuchen, auf das sechste Zeichen via~\pythonil{[5]} zuzugreifen?\uncover<28->{ Dann gibt es eine Fehlermeldung. Das geht nämlich nicht.%
}}%
%
\only<-30>{\item<29-> Wir können einen String auch \inQuotes{von hinten} indizieren. \pythonil{x[-1]} liefert das letzte Zeichen des Strings~\pythonil{x}.\uncover<30->{ Bei \inQuotes{Hello} ist das~\inQuotes{o}.%
}}%
%
\only<-32>{\item<31-> Das vorletze Zeichen bekommen wir durch Index~\pythonil{[-2]}.\uncover<32->{ Bei \inQuotes{Hello} ist das~\inQuotes{l}.%
}}%
%
\only<-34>{\item<33-> Das vor-vorletze Zeichen bekommen wir durch Index~\pythonil{[-3]}.\uncover<34->{ Bei \inQuotes{Hello} ist das~\inQuotes{l}.%
}}%
%
\only<-36>{\item<35-> Das vierte Zeichen von hinten bekommen wir durch Index~\pythonil{[-4]}.\uncover<36->{ Bei \inQuotes{Hello} ist das~\inQuotes{e}.%
}}%
%
\only<-38>{\item<37-> Das fünfte Zeichen von hinten bekommen wir durch Index~\pythonil{[-5]}.\uncover<36->{ Bei \inQuotes{Hello} ist das~\inQuotes{H}.%
}}%
%
\only<-40>{\item<39-> \inQuotes{Hello} hat nur fünf Zeichen. Was passiert, wenn wir versuchen, auf das sechste Zeichen von hinten zuzugreifen?\uncover<40->{ Dann gibt es wieder eine Fehlermeldung, weil das nämlich nicht geht.%
}}%
%
\only<-42>{\item<41-> Wir können auch ganze Substrings (Unterzeichenketten) extrahieren, in dem wir den  Index~\pythonil{i} des ersten Zeichens und den Index~\pythonil{j} \emph{nach} dem letzten zu extrahierenden Zeichen angeben als~\pythonil{[i:j]}.\uncover<42->{ \pythonil{[0:3]} ergibt also die Zeichen an den Indizes~0, 1, 2. Bei \inQuotes{Hello} ist das \pythonil{\"Hel\"}.%
}}%
%
\only<-44>{\item<43-> \pythonil{\"Hello\"[1:3]} ist der Substring, der beim zweiten Zeichen anfängt~(Index~1) und \alert{vor} dem vierten Zeichen~(Index~3) aufhört.\uncover<44->{ Also \pythonil{\"el\"}.%
}}%
%
\only<-46>{\item<45-> Lassen wir einfach den zweiten Index weg, dann werden alle Zeichen bis zum Ende des Strings zurückgegeben.\uncover<46->{ \pythonil{\"Hello\"[2:]} ergibt also die Zeichen an den Indices 2, 3, 4, und~5, also \pythonil{\"llo\"}.%
}}%
%
\only<-48>{\item<47-> Wir können genausogut auch negative Indizes verwenden, die dann wieder von hinten Zählen.\uncover<48->{ \pythonil{\"Hello\"[1:-2]} started an Index~1 und hört \alert{vor} dem zweiten Zeichen von hinten~(Index~3) auf. Die Indizes im Originalstring gehen von~0 bis~5, wir haben also die Zeichen an den Indizes von~1 bis~2, demach also \pythonil{\"el\"}.%
}}%
%
\only<-50>{\item<49-> Wir können auch den ersten Index weglassen, dann fängt der zurückgelieferte String am Anfang der Zeichenkette an und geht bis \alert{vor} den zweiten Index.\uncover<50->{ \pythonil{\"Hello\"[:-2]} fängt am Anfang an und hört \alert{vor} dem vorletzten Zeichen auf, ist also~\pythonil{\"Hel\"}.%
}}%
%
\only<-52>{\item<51-> Wir können auch drei Werte angeben, \pythonil{[i:j:k]}, wobei \pythonil{i} wieder der Index des ersten Zeichens und \pythonil{j} der Index \alert{nach} dem letzten Zeichen ist. \pythonil{k}~ist die Schrittweite.\uncover<51->{ \pythonil{\"Hello World!\"[1:8:2]} liefert \alert{jedes zweite} Zeichen beginnend an Index~1 und endent \alert{vor} Index~8, also die Zeichen an den Indices~1, 3, 5, und~7. Das sind~\pythonil{\"el o\"}.%
}}%
%
\only<53->{%
\item<53-> Damit haben wir also erstmal ein Grundverständnis, was Strings sind, wie wir sie verketten, ihre Länge bestimmen, und wieder auseinanderfummeln können.%
}%
%
\end{itemize}%
%
\locateGraphic{2}{width=0.8\paperwidth}{graphics/strBasic/strBasic01terminal}{0.1}{0.33}%
\locateGraphic{3}{width=0.8\paperwidth}{graphics/strBasic/strBasic02python3}{0.1}{0.33}%
\locateGraphic{4}{width=0.8\paperwidth}{graphics/strBasic/strBasic03python3done}{0.1}{0.33}%
\locateGraphic{5-8}{width=0.8\paperwidth}{graphics/strBasic/strBasic04dqHW}{0.1}{0.33}%
\locateGraphic{9}{width=0.8\paperwidth}{graphics/strBasic/strBasic05dqHWdone}{0.1}{0.33}%
\locateGraphic{10}{width=0.8\paperwidth}{graphics/strBasic/strBasic06sqHW}{0.1}{0.33}%
\locateGraphic{11}{width=0.8\paperwidth}{graphics/strBasic/strBasic07sqHWdone}{0.1}{0.33}%
%
\locate{12}{%
\parbox{0.9\paperwidth}{%
\bestPractice{strDoubleQuote}{%
Beim Definieren von String-Literalen sollte die Variante mit doppelten Anführungszeichen~(\pythonil{\"...\"}) bevorzugt werden.~(Der \citetitle{PEP8}\cite{PEP8} gibt keine Empfehlung, aber vielleicht für Konsistenz mit den~\citetitle{PEP257}\cite{PEP257}.)%
}}}{0.05}{0.53}%
%
\locateGraphic{13}{width=0.8\paperwidth}{graphics/strBasic/strBasic08strConcat}{0.1}{0.33}%
\locateGraphic{14}{width=0.8\paperwidth}{graphics/strBasic/strBasic09strConcatDone}{0.1}{0.33}%
\locateGraphic{15}{width=0.8\paperwidth}{graphics/strBasic/strBasic10strLen}{0.1}{0.33}%
\locateGraphic{16}{width=0.8\paperwidth}{graphics/strBasic/strBasic11strLenDone}{0.1}{0.33}%
\locateGraphic{17}{width=0.8\paperwidth}{graphics/strBasic/strBasic12Hello0}{0.1}{0.33}%
\locateGraphic{18}{width=0.8\paperwidth}{graphics/strBasic/strBasic13Hello0done}{0.1}{0.33}%
\locateGraphic{19}{width=0.8\paperwidth}{graphics/strBasic/strBasic14Hello1}{0.1}{0.33}%
\locateGraphic{20}{width=0.8\paperwidth}{graphics/strBasic/strBasic15Hello1done}{0.1}{0.33}%
\locateGraphic{21}{width=0.8\paperwidth}{graphics/strBasic/strBasic16Hello2}{0.1}{0.33}%
\locateGraphic{22}{width=0.8\paperwidth}{graphics/strBasic/strBasic17Hello2done}{0.1}{0.33}%
\locateGraphic{23}{width=0.8\paperwidth}{graphics/strBasic/strBasic18Hello3}{0.1}{0.33}%
\locateGraphic{24}{width=0.8\paperwidth}{graphics/strBasic/strBasic19Hello3done}{0.1}{0.33}%
\locateGraphic{25}{width=0.8\paperwidth}{graphics/strBasic/strBasic20Hello4}{0.1}{0.33}%
\locateGraphic{26}{width=0.8\paperwidth}{graphics/strBasic/strBasic21Hello4done}{0.1}{0.33}%
\locateGraphic{27}{width=0.8\paperwidth}{graphics/strBasic/strBasic22Hello5}{0.1}{0.33}%
\locateGraphic{28}{width=0.8\paperwidth}{graphics/strBasic/strBasic23Hello5done}{0.1}{0.33}%
\locateGraphic{29}{width=0.8\paperwidth}{graphics/strBasic/strBasic24HelloM1}{0.1}{0.33}%
\locateGraphic{30}{width=0.8\paperwidth}{graphics/strBasic/strBasic25HelloM1done}{0.1}{0.33}%
\locateGraphic{31}{width=0.8\paperwidth}{graphics/strBasic/strBasic26HelloM2}{0.1}{0.33}%
\locateGraphic{32}{width=0.8\paperwidth}{graphics/strBasic/strBasic27HelloM2done}{0.1}{0.33}%
\locateGraphic{33}{width=0.8\paperwidth}{graphics/strBasic/strBasic28HelloM3}{0.1}{0.33}%
\locateGraphic{34}{width=0.8\paperwidth}{graphics/strBasic/strBasic29HelloM3done}{0.1}{0.33}%
\locateGraphic{35}{width=0.8\paperwidth}{graphics/strBasic/strBasic30HelloM4}{0.1}{0.33}%
\locateGraphic{36}{width=0.8\paperwidth}{graphics/strBasic/strBasic31HelloM4done}{0.1}{0.33}%
\locateGraphic{37}{width=0.8\paperwidth}{graphics/strBasic/strBasic32HelloM5}{0.1}{0.33}%
\locateGraphic{38}{width=0.8\paperwidth}{graphics/strBasic/strBasic33HelloM5done}{0.1}{0.33}%
\locateGraphic{39}{width=0.8\paperwidth}{graphics/strBasic/strBasic34HelloM6}{0.1}{0.33}%
\locateGraphic{40}{width=0.8\paperwidth}{graphics/strBasic/strBasic35HelloM6done}{0.1}{0.33}%
\locateGraphic{41}{width=0.8\paperwidth}{graphics/strBasic/strBasic36Hello0C3}{0.1}{0.33}%
\locateGraphic{42}{width=0.8\paperwidth}{graphics/strBasic/strBasic37Hello0C3done}{0.1}{0.33}%
\locateGraphic{43}{width=0.8\paperwidth}{graphics/strBasic/strBasic38Hello1C3}{0.1}{0.33}%
\locateGraphic{44}{width=0.8\paperwidth}{graphics/strBasic/strBasic39Hello1C3done}{0.1}{0.33}%
\locateGraphic{45}{width=0.8\paperwidth}{graphics/strBasic/strBasic40Hello2C}{0.1}{0.33}%
\locateGraphic{46}{width=0.8\paperwidth}{graphics/strBasic/strBasic41Hello2Cdone}{0.1}{0.33}%
\locateGraphic{47}{width=0.8\paperwidth}{graphics/strBasic/strBasic42Hello1Cm2}{0.1}{0.33}%
\locateGraphic{48}{width=0.8\paperwidth}{graphics/strBasic/strBasic43Hello1Cm2done}{0.1}{0.33}%
\locateGraphic{49}{width=0.8\paperwidth}{graphics/strBasic/strBasic44HelloCm2}{0.1}{0.33}%
\locateGraphic{50}{width=0.8\paperwidth}{graphics/strBasic/strBasic45HelloCm2done}{0.1}{0.33}%
\locateGraphic{51}{width=0.8\paperwidth}{graphics/strBasic/strBasic46HelloWorld1C8C2}{0.1}{0.33}%
\locateGraphic{52}{width=0.8\paperwidth}{graphics/strBasic/strBasic47HelloWorld1C8C2done}{0.1}{0.33}%
%
\end{frame}%
%
\begin{frame}[t]%
\frametitle{Weitere Grundlegende String Operationen}%
\begin{itemize}%
%
\only<-1,53->{\item Schauen wir uns die ein paar weitere grundlegenden Operationen für Strings an.}%
%
\only<-3>{\item<2-> Mit dem Operator \pythonil{a in b} prüfen wir, ob die Zeichenkette~\pythonil{a} irgendwo im String~\pythonil{b} enthalten ist.\uncover<3->{ Das ist hier der Fall: \pythonil{\"World\"} ist tatsächlich im String~\pythonil{\"Hello World!\"} enthalten.%
}}%
%
\only<-5>{\item<4-> Ist \pythonil{\"Earth\"} irgendwo in~\pythonil{\"Hello World!\"} enthalten?\uncover<5->{ Nein.%
}}%
%
\only<-7>{\item<6-> Die Funktion~\pythonil{a.find(b)} sucht den Index, an dem die Zeichenkette \pythonil{b} in \pythonil{a} beginnt.\uncover<7->{ \pythonil{\"World\"} beginnt an Index~6 in \pythonil{\"Hello World!\"}.%
}}%
%
\only<-12>{\item<8-> \only<-10>{String-Funktionen und Vergleiche sind \inQuotes{case-sensitive}:~}Groß- und Kleinbuchstaben werden als unterschiedlich betrachtet.\uncover<9->{\only<-11>{ %
Somit gilt \pythonil{\"W\" != \"w\"}.}\uncover<10->{ %
Somit kann \pythonil{\"world\"} nicht in \pythonil{\"Hello World!\"} gefunden werden.\uncover<11->{ %
Somit liefert die Function~\pythonil{-1} zurück.\uncover<12->{ %
Beachte also:~Niemals das Ergebnis von \pythonil{find} direkt zum Indizieren nehmen, denn \pythonil{-1} steht für \inQuotes{letztes Zeichen}\dots%
}}}}}%
%
\only<-14>{\item<13-> Wo befindet sich~\pythonil{\"l\"} in \pythonil{\"Hello World!\"}?\uncover<14->{ An Index~2.%
}}%
%
\only<-16>{\item<15-> Wir können auch den Index angeben, ab dem gesucht werden soll:~Wo befindet sich~\pythonil{\"l\"} in \pythonil{\"Hello World!\"} \emph{wenn wir ab Index~3 suchen}?\uncover<16->{ An Index~3.%
}}%
%
\only<-18>{\item<17-> Wo befindet sich~\pythonil{\"l\"} in \pythonil{\"Hello World!\"} wenn wir ab Index~4 suchen?\uncover<18->{ An Index~9.%
}}%
%
\only<-20>{\item<19-> Wo befindet sich~\pythonil{\"l\"} in \pythonil{\"Hello World!\"} wenn wir ab Index~10 suchen?\uncover<20->{ Dann finden wir kein \inQuotes{l} mehr und \pythonil{-1} wird zurückgegeben.%
}}%
%
\only<-22>{\item<21-> \pythonil{rfind} sucht von hinten/rechts nach vorne.\uncover<22->{ Von rechts aus gehen findet sich das erste Auftreten von \pythonil{\"l\"} in \pythonil{\"Hello World!\"} an Index~9.%
}}%
%
\only<-24>{\item<23-> Sowohl bei \pythonil{find} als auch bei \pythonil{rfind} können wir keinen Index, den Startindex, oder den Start- und den (exklusiven) End-Index für die Suche angeben. Wir suchen nun in der Zeichekette von Index~2 bis \alert{vor} Index~9 von rechts.\uncover<24->{ Und finden \pythonil{\"l\"} an Index~3.%
}}%
%
\only<-26>{\item<25-> Wir suchen nun in der Zeichekette von Index~0 bis \alert{vor} Index~3 von rechts.\uncover<26->{ Und finden \pythonil{\"l\"} an Index~2.%
}}%
%
\only<-28>{\item<27-> Wir suchen nun in der Zeichekette von Index~0 bis \alert{vor} Index~2 von rechts.\uncover<28->{ Und finden \pythonil{\"l\"} gar nicht mehr, bekommen also \pythonil{-1} zurück.%
}}%
%
\only<-30>{\item<29-> \pythonil{a.replace(b, c)} ersetzt alle Auftreten von \pythonil{b} in \pythonil{a} mit \pythonil{c} und gibt das Ergebnis als neuen String zurück\uncover<30->{ Ersetzen wir alle \pythonil{\"Hello\"} in \pythonil{\"Hello World!\"} mit \pythonil{\"Hi\"}, so bekommen wir \pythonil{\"Hi World!\"}.%
}}%
%
\only<-32>{\item<31-> Ersetzen wir alle \pythonil{\"Hello\"} in \pythonil{\"Hello World! Hello!\"} mit \pythonil{\"Hi\"}\only<-31>{\dots}\uncover<32->{, so bekommen wir \pythonil{\"Hi World! Hi!\"}.%
}}%
%
\only<-34>{\item<33-> Ersetzen wir alle \pythonil{\"Hello\"} in \pythonil{\"Hello World!\"} mit \pythonil{\"Hello! Hello!\"}\only<-33>{\dots}\uncover<34->{, so bekommen wir \pythonil{\"Hello! Hello! World!\"}. %
Das Ersetzen funktioniert also nicht rekursiv, ersetzt also nicht in bereits ersetzten Strings.%
}}%
%
\only<-36>{\item<35-> \pythonil{a.strip()} entfernt alle so-genannten \inQuotes{whitespace}-Zeichen~(Leerzeichen, Newlines, Tabs) am Anfang und Ende eines Strings und gibt das Ergebnis als neuen String zurück\uncover<36->{ \pythonil{\" Hello World! \"} wird so zu \pythonil{\"Hello World!\"}.%
}}%
%
\only<-38>{\item<37-> \pythonil{a.lstrip()} entfernt alle so-genannten \inQuotes{whitespace}-Zeichen~(Leerzeichen, Newlines, Tabs) am Anfang eines Strings und gibt das Ergebnis als neuen String zurück\uncover<38->{ \pythonil{\" Hello World! \"} wird so zu \pythonil{\"Hello World! \"}.%
}}%
%
\only<-40>{\item<39-> \pythonil{a.rstrip()} entfernt alle so-genannten \inQuotes{whitespace}-Zeichen~(Leerzeichen, Newlines, Tabs) am Ende eines Strings und gibt das Ergebnis als neuen String zurück\uncover<40->{ \pythonil{\" Hello World! \"} wird so zu \pythonil{\" Hello World!\"}.%
}}%
%
\only<-42>{\item<41-> \pythonil{a.lower()} wandelt alle Großbuchstaben in \pythonil{a} in Kleinbuchstaben um und gibt das Ergebnis als neuen String zurück.\uncover<42->{ \pythonil{\"Hello World!\"} wird so zu \pythonil{\"hello world!\"}.%
}}%
%
\only<-44>{\item<43-> \pythonil{a.upper()} wandelt alle Kleinbuchstaben in \pythonil{a} in Großbuchstaben um und gibt das Ergebnis als neuen String zurück.\uncover<44->{ \pythonil{\"Hello World!\"} wird so zu \pythonil{\"HELLOW WORLD\"}.%
}}%
%
\only<-46>{\item<45-> \pythonil{a.startswith(b)} gibt \pythonil{True} zurück wenn und nur wenn \pythonil{a} mit \pythonil{b} anfängt.\uncover<46->{ Und weil es case-sensitive ist, ergibt das hier \pythonil{False}.%
}}%
%
\only<-48>{\item<47-> \pythonil{a.startswith(b)} gibt \pythonil{True} zurück wenn und nur wenn \pythonil{a} mit \pythonil{b} anfängt.\uncover<48->{ Aber jetzt stimmt es.%
}}%
%
\only<-50>{\item<49-> \pythonil{a.endswith(b)} gibt \pythonil{True} zurück wenn und nur wenn \pythonil{a} mit \pythonil{b} endet.\uncover<50->{ Und das ist hier natürlich nicht der Fall.%
}}%
%
\only<-52>{\item<51-> \pythonil{a.endswith(b)} gibt \pythonil{True} zurück wenn und nur wenn \pythonil{a} mit \pythonil{b} endet.\uncover<52->{ Aber das stimmt.%
}}%
%
\item<53-> So, nun haben wir schon ziemlich viele String-Funktionen gelernt.%
\end{itemize}%
%
\locateGraphic{1}{width=0.8\paperwidth}{graphics/strBasic/strBasic03python3done}{0.1}{0.33}%
\locateGraphic{2}{width=0.8\paperwidth}{graphics/strOp/strOp01inYes}{0.1}{0.33}%
\locateGraphic{3}{width=0.8\paperwidth}{graphics/strOp/strOp02inYesDone}{0.1}{0.33}%
\locateGraphic{4}{width=0.8\paperwidth}{graphics/strOp/strOp03inNo}{0.1}{0.33}%
\locateGraphic{5}{width=0.8\paperwidth}{graphics/strOp/strOp04inNoDone}{0.1}{0.33}%
\locateGraphic{6}{width=0.8\paperwidth}{graphics/strOp/strOp05findWorld}{0.1}{0.33}%
\locateGraphic{7}{width=0.8\paperwidth}{graphics/strOp/strOp06findWorldDone}{0.1}{0.33}%
\locateGraphic{8-9}{width=0.8\paperwidth}{graphics/strOp/strOp07findworld}{0.1}{0.33}%
\locateGraphic{10-12}{width=0.8\paperwidth}{graphics/strOp/strOp08findworldDone}{0.1}{0.33}%
\locateGraphic{13}{width=0.8\paperwidth}{graphics/strOp/strOp09findl}{0.1}{0.33}%
\locateGraphic{14}{width=0.8\paperwidth}{graphics/strOp/strOp10findlDone}{0.1}{0.33}%
\locateGraphic{15}{width=0.8\paperwidth}{graphics/strOp/strOp11findl3}{0.1}{0.33}%
\locateGraphic{16}{width=0.8\paperwidth}{graphics/strOp/strOp12findl3done}{0.1}{0.33}%
\locateGraphic{17}{width=0.8\paperwidth}{graphics/strOp/strOp13findl4}{0.1}{0.33}%
\locateGraphic{18}{width=0.8\paperwidth}{graphics/strOp/strOp14findl4done}{0.1}{0.33}%
\locateGraphic{19}{width=0.8\paperwidth}{graphics/strOp/strOp15findl10}{0.1}{0.33}%
\locateGraphic{20}{width=0.8\paperwidth}{graphics/strOp/strOp16findl10done}{0.1}{0.33}%
\locateGraphic{21}{width=0.8\paperwidth}{graphics/strOp/strOp17rfindl}{0.1}{0.33}%
\locateGraphic{22}{width=0.8\paperwidth}{graphics/strOp/strOp18rfindlDone}{0.1}{0.33}%
\locateGraphic{23}{width=0.8\paperwidth}{graphics/strOp/strOp19rfindl2c9}{0.1}{0.33}%
\locateGraphic{24}{width=0.8\paperwidth}{graphics/strOp/strOp20rfindl2c9done}{0.1}{0.33}%
\locateGraphic{25}{width=0.8\paperwidth}{graphics/strOp/strOp21rfindl0c3}{0.1}{0.33}%
\locateGraphic{26}{width=0.8\paperwidth}{graphics/strOp/strOp22rfindl0c3done}{0.1}{0.33}%
\locateGraphic{27}{width=0.8\paperwidth}{graphics/strOp/strOp23rfindl0c2}{0.1}{0.33}%
\locateGraphic{28}{width=0.8\paperwidth}{graphics/strOp/strOp24rfindl0c2done}{0.1}{0.33}%
\locateGraphic{29}{width=0.8\paperwidth}{graphics/strOp/strOp25replaceHelloHi}{0.1}{0.33}%
\locateGraphic{30}{width=0.8\paperwidth}{graphics/strOp/strOp26replaceHelloHiDone}{0.1}{0.33}%
\locateGraphic{31}{width=0.8\paperwidth}{graphics/strOp/strOp27replaceHelloHi2}{0.1}{0.33}%
\locateGraphic{32}{width=0.8\paperwidth}{graphics/strOp/strOp28replaceHelloHi2done}{0.1}{0.33}%
\locateGraphic{33}{width=0.8\paperwidth}{graphics/strOp/strOp29replaceHelloHelloHello}{0.1}{0.33}%
\locateGraphic{34}{width=0.8\paperwidth}{graphics/strOp/strOp30replaceHelloHelloHelloDone}{0.1}{0.33}%
\locateGraphic{35}{width=0.8\paperwidth}{graphics/strOp/strOp31strip}{0.1}{0.33}%
\locateGraphic{36}{width=0.8\paperwidth}{graphics/strOp/strOp32stripDone}{0.1}{0.33}%
\locateGraphic{37}{width=0.8\paperwidth}{graphics/strOp/strOp33lstrip}{0.1}{0.33}%
\locateGraphic{38}{width=0.8\paperwidth}{graphics/strOp/strOp34lstripDone}{0.1}{0.33}%
\locateGraphic{39}{width=0.8\paperwidth}{graphics/strOp/strOp35rstrip}{0.1}{0.33}%
\locateGraphic{40}{width=0.8\paperwidth}{graphics/strOp/strOp36rstripDone}{0.1}{0.33}%
\locateGraphic{41}{width=0.8\paperwidth}{graphics/strOp/strOp37lower}{0.1}{0.33}%
\locateGraphic{42}{width=0.8\paperwidth}{graphics/strOp/strOp38lowerDone}{0.1}{0.33}%
\locateGraphic{43}{width=0.8\paperwidth}{graphics/strOp/strOp39upper}{0.1}{0.33}%
\locateGraphic{44}{width=0.8\paperwidth}{graphics/strOp/strOp40upperDone}{0.1}{0.33}%
\locateGraphic{45}{width=0.8\paperwidth}{graphics/strOp/strOp41startswithNo}{0.1}{0.33}%
\locateGraphic{46}{width=0.8\paperwidth}{graphics/strOp/strOp42startswithNoDone}{0.1}{0.33}%
\locateGraphic{47}{width=0.8\paperwidth}{graphics/strOp/strOp43startswithYes}{0.1}{0.33}%
\locateGraphic{48}{width=0.8\paperwidth}{graphics/strOp/strOp44startswithYesDone}{0.1}{0.33}%
\locateGraphic{49}{width=0.8\paperwidth}{graphics/strOp/strOp45endswithNo}{0.1}{0.33}%
\locateGraphic{50}{width=0.8\paperwidth}{graphics/strOp/strOp46endswithNoDone}{0.1}{0.33}%
\locateGraphic{51}{width=0.8\paperwidth}{graphics/strOp/strOp47endswithYes}{0.1}{0.33}%
\locateGraphic{52}{width=0.8\paperwidth}{graphics/strOp/strOp48endswithYesDone}{0.1}{0.33}%
\end{frame}%
%
\section{Die Funktion str und f-Strings}%
%
\begin{frame}[t]%
\frametitle{Die Funktion str}%
\begin{itemize}%
\only<-2>{%
\item Für Werte~\pythonil{a} der meisten Datentypen liefert die Funktion \pythonil{str(a)} einen entsprechenden String zurück.%
\item<2-> Probieren wir das mal aus.%
}%
%
\only<-4>{\item<3-> Wenden wir die Funktion \pythonil{str} auf den \pythonil{int}\nobreakdashes-Wert~\pythonil{23} an\only<-3>{\dots}\uncover<4->{ so liefert sie den \pythonil{str}\nobreakdashes-Wert~\pythonil{\"23\"}.%
}}%
%
\only<-6>{\item<5-> Wenden wir die Funktion \pythonil{str} auf den \pythonil{float}\nobreakdashes-Wert~\pythonil{23.5} an\only<-5>{\dots}\uncover<6->{ so liefert sie den \pythonil{str}\nobreakdashes-Wert~\pythonil{\"23.5\"}.%
}}%
%
\only<-8>{\item<7-> Wenn der Ausdruck \pythonil{str(4) + str(5)} ausgerechnet wird\only<-7>{\dots}\uncover<8->{, dann werden zuerst die beiden \pythonil{str}\nobreakdashes-Funktionen ausgerechnet, was dann \pythonil{\"4\" + \"5\"} ergibt. Dies ergibt dann wiederum~\pythonil{\"45\"}.%
}}%
%
\only<-10>{\item<9-> Wir können nahezu beliebige Aufrufe von \pythonil{str(...)} mit anderen Strings und \pythonil{+} kombinieren.\uncover<10->{ Beispielsweise ist \pythonil{str(1 < 5)} das selbe wie \pythonil{str(True)}, was \pythonil{\"True\"} ergibt. \pythonil{str(1 + 5)} ist das selbe wie \pythonil{str(6)}, was dann \pythonil{\"6\"} ergibt. Alle Strings zusammen ergeben dann\dots%
}}%
%
\end{itemize}%
%
\locateGraphic{2}{width=0.8\paperwidth}{graphics/strBasic/strBasic03python3done}{0.1}{0.33}%
\locateGraphic{3}{width=0.8\paperwidth}{graphics/toStr/toStr01str23}{0.1}{0.33}%
\locateGraphic{4}{width=0.8\paperwidth}{graphics/toStr/toStr02str23done}{0.1}{0.33}%
\locateGraphic{5}{width=0.8\paperwidth}{graphics/toStr/toStr03str23d5}{0.1}{0.33}%
\locateGraphic{6}{width=0.8\paperwidth}{graphics/toStr/toStr04str23d5done}{0.1}{0.33}%
\locateGraphic{7}{width=0.8\paperwidth}{graphics/toStr/toStr05strPlusStr}{0.1}{0.33}%
\locateGraphic{8}{width=0.8\paperwidth}{graphics/toStr/toStr06strPlusStrDone}{0.1}{0.33}%
\locateGraphic{9}{width=0.8\paperwidth}{graphics/toStr/toStr07multiConvConc}{0.1}{0.33}%
\locateGraphic{10}{width=0.8\paperwidth}{graphics/toStr/toStr08multiConvConcDone}{0.1}{0.33}%
%
\end{frame}%
%
\begin{frame}[t]%
\frametitle{f-strings}%
\begin{itemize}%
%
\only<-10,50->{%
\only<-10>{%
\item \pythonil{str(1 < 5) + \" is True and \" + str(1 + 5) + \" = 6.\"} {\dots} das sieht ziemlich umständlich aus.}%
\item<2-> In der Tat gibt es eine Möglichkeit, schön formatierte Zeichenketten zu erstellen, die Ergebnisse von Berechnungen beinhalten können\only<-2>{.}\uncover<3->{:~\glslink{fstring}{f\nobreakdashes-Strings}.}%
\item<4-> \glslink{fstring}{f\nobreakdashes-Strings} erlauben es uns, die Ergebnisse von Ausdrücken in Zeichenketten einzufügen\cite{PEP498,PSF:P3D:TPT:FSL,M2017WAFSIPAHCIUT,B2023PFS,DEVC:WFAAPODSCMIP,G2025MFIP}.%
\item<5-> Sie haben eine einfache Syntax\only<-5>{.}\uncover<6->{:%
\begin{enumerate}%
\item Sie sind mit \pythonil{f\"...\"} begrenzt~(beachten Sie das kleine \textil{f} vor dem einleitenden Anführungszeichen!).%
\item<7-> Sie können beliebige Ausdrücke enthalten, die jeweils mit \pythonil{\{\...\}} begrenzt sind.%
\item<8-> Zusätzlich können Markierungen für die Formatierung der Ausdruck-Ergebnisse angegeben werden.%
\item<9-> Während der sogenannten \glslink{strinterpolation}{String\nobreakdashes-Interpolation}, welche automatisch passiert, werden diese Ausdrücke dann ausgerechnet und ihre Ergebnisse an ihrer Stelle in den Text eingefügt.%
\end{enumerate}%
}%
\only<-10>{%
\item<10-> So wird aus \pythonil{f\"a\{6-1\}b\"} dann \pythonil{\"a5b\"}.}%
}%
%
\only<-11>{%
\item<11-> Probieren wir das mal aus.%
}%
%
\only<-13>{\item<12-> Eine \pythonil{int}-Zahl ist ein Ausdruck. Schreiben wir also mal einen \glslink{fstring}{f\nobreakdashes-String}, der eine Zahl in einen Text umwandelt und mit anderem Text kombiniert.\uncover<13->{ Das funktioniert problemlos.%
}}%
%
\only<-15>{\item<14-> \glslink{fstring}{f\nobreakdashes-Strings} erlauben es uns auch, die Erbenisse von Ausdrücken zu formatieren. Schreiben wir z.B.\ \textil{:,} hinten an den Ausdruck (hier: der Zahl) ran, dann wird der Ausdruck als Zahl behandelt und es werden Tausender-Trennzeichen~(hier: Kommas) bei der Umwandlung in Text eingefügt.\uncover<15->{ Aus der Zahl \textil{12345678901234} wird der Text~\textil{12,345,678,901,234}.%%
}}%
%
\only<-17>{\item<16-> Wir haben in den Slides zum Datentyp \pythonil{int} verschiedene Zahlensysteme erwähnt. Fügen wir \textil{:x} an den Ausdruck an, so wird sein Ergebnis als Hexadezimalzahl dargestellt.\uncover<17->{ So wird aus der (Dezimal-)Zahl \textil{12345678901234} die Hexadezimalzahl \textil{b3a73ce2ff2}.%
}}%
%
\only<-19>{\item<18-> Wollen wir zusätzlich noch das \textil{0x}\nobreakdashes-Präfix mit davor haben, schreiben wir \textil{:\#x} anstatt von \textil{:x}.\uncover<19->{ So wird aus der (Dezimal-)Zahl \textil{12345678901234} die Hexadezimalzahl \textil{0xb3a73ce2ff2}.%
}}%
%
\only<-21>{\item<20-> Fügen wir \textil{:b} an den Ausdruck an, so wird sein Ergebnis als Binärzahl dargestellt.\uncover<21->{ So wird aus der (Dezimal-)Zahl \textil{1234567890} die Binärzahl \textil{1001001100101100000001011010010}.%
}}%
%
\only<-23>{\item<22-> Fügen wir \textil{:\#b} an den Ausdruck an, so wird sein Ergebnis als Binärzahl mit dem Präfix~\textil{0b} dargestellt.\uncover<23->{ So wird aus der (Dezimal-)Zahl \textil{1234567890} die Binärzahl \textil{0b1001001100101100000001011010010}.%
}}%
%
\only<-25>{\item<24-> Nun sind \pythonil{int}-Zahlen zwar Ausdrücke, aber nicht besonders spannend. Probieren wir es diesmal mit einer echten kleinen Berechnung.\uncover<25->{ \pythonil{f\"\{5 + 4\}\"} wird tatsächlich zu~\pythonil{\"9\"}.%
}}%
%
\only<-27>{\item<26-> Wir können in \glslink{fstring}{f\nobreakdashes-Strings} auf \alert{alle} Werte zugreifen, auf die wir aktuell Zugriff haben. Importieren wir also interessehalber mal wieder \numberPi\ from \pythonil{math}\nobreakdashes-Modul.\uncover<27->{ Was natürlich klappt.%
}}%
%
\only<-29>{\item<28-> Nun können wir auch \pythonil{pi} in \glslink{fstring}{f\nobreakdashes-Strings} verwenden.\uncover<29->{ Und sein Wert wird eingefügt.%
}}%
%
\only<-31>{\item<30-> Wenn wir alle Stellen der \pythonil{float}\nobreakdashes-Konstante \numberPi\ drucken, dann ist das ziemlich viel. Wollen wir nur \pythonil{NNN} Nachkommastellen einer Zahl angezeigt bekommen, so fügen wir einfach \textil{:.NNNf} an den Ausdruck an.\uncover<31->{ Das \textil{:.2f} bewirkt, dass der \pythonil{float}\nobreakdashes-Wert \pythonil{pi} auf zwei Nachkommastellen gerundet wird.%
}}%
%
\only<-33>{\item<32-> Wir können eine Zahl in Prozent konvertieren, in dem wir \textil{:\%} an den Ausdruck anhängen. Wollen wir den Prozentwert mit \textil{NNN} Nachkommastellen, so hängen wir stattdessen~\textil{:.NNN\%} an.\uncover<33->{ $\frac{1}{321}\approx0.00311526$, was $0.311526\%$~entspricht. Die Ausgabe stimmt also.%
}}%
%
\only<-35>{\item<34-> Wir können auch Tausender-Trennzeichen und Nachkommastellenangaben kombinieren.\uncover<35->{ \pythonil{1.2345533e4} entspricht $1.2345533*10^4$, was wiederum $12345.533$~ist. Auf eine Nachkommastelle genau ergibt das $12345.5$.%
}}%
%
\only<-37>{\item<36-> Importieren wir die Sinus\nobreakdashes-Funktion aus dem Modul~\pythonil{math}.\uncover<37->{ Gar kein Problem.%
}}%
%
\only<-39>{\item<38-> Und schon können wir sie in \glslink{fstring}{f\nobreakdashes-Strings} verwenden, deren Ausgabe wir wieder beliebig formatieren können.\uncover<39->{ $\sin{\frac{\numberPi}{4}}=\frac{\sqrt{2}}{2}\approx0.7071067811865476$.%
}}%
%
\only<-41>{\item<40-> Das Format \textil{:e} druckt eine Zahl in der wissenschaftlichen Notation. \textil{:.DDDg} benutzt ebenfalls die wissenschaftliche Notation, druckt jedoch nur \textil{DDD}~Ziffern~(insgesamt, nicht Nachkommastellen).\uncover<41->{ Aus \textil{:.3g} folgert, dass nur drei Ziffern~(\textil{1.24}) gedruckt werden.%
}}%
%
\only<-43>{\item<42-> Wenn die geschwungenen Klammern~\textil{\{} und \textil{\}} Ausdrücke in den Zeichenketten begrenzen {\dots} was machen wir dann, wenn wir mal eine geschwungene Klammer in einem String haben wollen?\uncover<43->{ Einfach \emph{zwei} geschwungene Klammern schreiben.%
}}%
%
\only<-45>{\item<44-> Ein wirklich cooles Feature ist, dass wir sowohl einen Ausdruck~\pythonil{A} als auch dessen Ergebnis~\pythonil{A} ausdrucken können, in dem wir \pythonil{f\"\{A = \}\"} schreiben.\uncover<45->{ Das wird dann in~\pythonil{\"A = E\"} umgewandelt.%
}}%
%
\only<-47>{\item<46-> Das funktioniert auch mit komplizierten Ausdrücken.\uncover<47->{ Problemlos.%
}}%
%
\only<-49>{\item<48-> Kommen wir nun zu unserem Anfangsbeispiel, dem komplizierten String \pythonil{str(1 < 5) + \" is True and \" + str(1 + 5) + \" = 6.\"}, zurück. Als \glslink{fstring}{f\nobreakdashes-String} sieht das so aus: \pythonil{f\"\{1 < 5\} is True and \{1 + 5\} = 6.\"}\uncover<47->{ Beide Varianten ergeben \pythonil{\"True is True and 6 = 6.\"}.%
}}%
%
\item<50-> Das waren natürlich nur einige Beispiele.%
\item<51-> Mehr Informationen gibt es unter %
\citetitle{B2023PFS}\cite{B2023PFS}, %
\citetitle{PSF:P3D:TPT:FSL}\cite{PSF:P3D:TPT:FSL}, und %
\citetitle{PEP498}\cite{PEP498}.%
%
\end{itemize}%
%
\locateGraphic{11}{width=0.8\paperwidth}{graphics/strBasic/strBasic03python3done}{0.1}{0.33}%
\locateGraphic{12}{width=0.8\paperwidth}{graphics/fstrings/fstrings01bigInt}{0.1}{0.33}%
\locateGraphic{13}{width=0.8\paperwidth}{graphics/fstrings/fstrings02bigIntDone}{0.1}{0.33}%
\locateGraphic{14}{width=0.8\paperwidth}{graphics/fstrings/fstrings03bigIntTsep}{0.1}{0.33}%
\locateGraphic{15}{width=0.8\paperwidth}{graphics/fstrings/fstrings04bigIntTsepDone}{0.1}{0.33}%
\locateGraphic{16}{width=0.8\paperwidth}{graphics/fstrings/fstrings05bigIntHex}{0.1}{0.33}%
\locateGraphic{17}{width=0.8\paperwidth}{graphics/fstrings/fstrings06bigIntHexDone}{0.1}{0.33}%
\locateGraphic{18}{width=0.8\paperwidth}{graphics/fstrings/fstrings07bigInt0x}{0.1}{0.33}%
\locateGraphic{19}{width=0.8\paperwidth}{graphics/fstrings/fstrings08bigInt0xDone}{0.1}{0.33}%
\locateGraphic{20}{width=0.8\paperwidth}{graphics/fstrings/fstrings09intBin}{0.1}{0.33}%
\locateGraphic{21}{width=0.8\paperwidth}{graphics/fstrings/fstrings10intBinDone}{0.1}{0.33}%
\locateGraphic{22}{width=0.8\paperwidth}{graphics/fstrings/fstrings11int0b}{0.1}{0.33}%
\locateGraphic{23}{width=0.8\paperwidth}{graphics/fstrings/fstrings12int0bDone}{0.1}{0.33}%
\locateGraphic{24}{width=0.8\paperwidth}{graphics/fstrings/fstrings13int5p4}{0.1}{0.33}%
\locateGraphic{25}{width=0.8\paperwidth}{graphics/fstrings/fstrings14int5p4done}{0.1}{0.33}%
\locateGraphic{26}{width=0.8\paperwidth}{graphics/fstrings/fstrings15importPi}{0.1}{0.33}%
\locateGraphic{27}{width=0.8\paperwidth}{graphics/fstrings/fstrings16importPiDone}{0.1}{0.33}%
\locateGraphic{28}{width=0.8\paperwidth}{graphics/fstrings/fstrings17pi}{0.1}{0.33}%
\locateGraphic{29}{width=0.8\paperwidth}{graphics/fstrings/fstrings18piDone}{0.1}{0.33}%
\locateGraphic{30}{width=0.8\paperwidth}{graphics/fstrings/fstrings19pi2dec}{0.1}{0.33}%
\locateGraphic{31}{width=0.8\paperwidth}{graphics/fstrings/fstrings20pi2decDone}{0.1}{0.33}%
\locateGraphic{32}{width=0.8\paperwidth}{graphics/fstrings/fstrings21c1d321perc2f}{0.1}{0.33}%
\locateGraphic{33}{width=0.8\paperwidth}{graphics/fstrings/fstrings22c1d321perc2fDone}{0.1}{0.33}%
\locateGraphic{34}{width=0.8\paperwidth}{graphics/fstrings/fstrings23bigWts2f}{0.1}{0.33}%
\locateGraphic{35}{width=0.8\paperwidth}{graphics/fstrings/fstrings24bigWts2fDone}{0.1}{0.33}%
\locateGraphic{36}{width=0.8\paperwidth}{graphics/fstrings/fstrings25importSin}{0.1}{0.33}%
\locateGraphic{37}{width=0.8\paperwidth}{graphics/fstrings/fstrings26importSinDone}{0.1}{0.33}%
\locateGraphic{38}{width=0.8\paperwidth}{graphics/fstrings/fstrings27sin5f}{0.1}{0.33}%
\locateGraphic{39}{width=0.8\paperwidth}{graphics/fstrings/fstrings28sin5fdone}{0.1}{0.33}%
\locateGraphic{40}{width=0.8\paperwidth}{graphics/fstrings/fstrings29eAndG}{0.1}{0.33}%
\locateGraphic{41}{width=0.8\paperwidth}{graphics/fstrings/fstrings30eAndGdone}{0.1}{0.33}%
\locateGraphic{42}{width=0.8\paperwidth}{graphics/fstrings/fstrings31braces}{0.1}{0.33}%
\locateGraphic{43}{width=0.8\paperwidth}{graphics/fstrings/fstrings32bracesDone}{0.1}{0.33}%
\locateGraphic{44}{width=0.8\paperwidth}{graphics/fstrings/fstrings33expr5p4}{0.1}{0.33}%
\locateGraphic{45}{width=0.8\paperwidth}{graphics/fstrings/fstrings34expr5p4done}{0.1}{0.33}%
\locateGraphic{46}{width=0.8\paperwidth}{graphics/fstrings/fstrings35exprSin}{0.1}{0.33}%
\locateGraphic{47}{width=0.8\paperwidth}{graphics/fstrings/fstrings36exprSinDone}{0.1}{0.33}%
\locateGraphic{48}{width=0.8\paperwidth}{graphics/fstrings/fstrings37oldExample}{0.1}{0.33}%
\locateGraphic{49}{width=0.8\paperwidth}{graphics/fstrings/fstrings38oldExampleDone}{0.1}{0.33}%
\end{frame}%
%
\section{Strings in Andere Datentypen Konvertieren}%
%
\begin{frame}[t]%
\frametitle{Strings in Andere Datentypen Konvertieren}%
\begin{itemize}%
\only<-6,26->{%
\item Oft wollen wir Daten in unser Program einlesen oder eingeben.%
\item<2-> Die Daten sind of als Text gespeichert oder werden als Text in ein \glslink{terminal}{Terminal} eingegeben.%
\item<3-> Wir brauchen sie dann aber oft als andere Datentypen, z.B.\ als Zahlen.%
\item<4-> Viele Datentypen in \python\ bieten eine Funktion an, die eine Zeichenkette in sie umwandelt.%
\only<-6>{%
\item<5-> \pythonil{int(\"4\")} erzeugt z.B.\ den \pythonil{int}\nobreakdashes-Wert~\pythonil{4}.%
}%
}%
\only<-7>{%
\item<6-> Schauen wir uns das mal an.%
}%
%
\only<-9>{\item<8-> Die Funktion \pythonil{int(a)} wandelt einen Wert~\pythonil{a} eines anderen Datentypen in einen \pythonil{int}\nobreakdashes-Wert um. \pythonil{a}~könnte z.B.\ ein String sein.\uncover<9->{ \pythonil{int(\"1111\")} ergibt daher den Integer~\pythonil{1111}.%
}}%
%
\only<-11>{\item<10-> Die Funktion hat einen optionalen zweiten Parameter, der die Basis des zu benutzenden Zahlensystems angibt. Die bekannten Präfixe werden automatisch erkannt.\uncover<11->{ \pythonil{int(\"0x1111\", 16)} wandelt eine Hexdezimalzahl \textil{0x1111} um, es gilt $1*1+1*16+1*16^2+1*16^3=1+16+256+4096=4396$ und das ergibt daher den Integer~\pythonil{4369}.%
}}%
%
\only<-13>{\item<12-> Die Funktion hat einen optionalen zweiten Parameter, der die Basis des zu benutzenden Zahlensystems angibt. Die bekannten Präfixe werden automatisch erkannt.\uncover<13->{ \pythonil{int(\"0b1111\", 2)} wandelt eine Binärzahl \textil{0b1111} um, es gilt $1*1+1*2+1*2^2+1*2^3=1+2+4+8=15$ und das ergibt daher den Integer~\pythonil{15}.%
}}%
%
\only<-15>{\item<14-> Die Funktion~\pythonil{float} wandelt einen Wert~\pythonil{a} eines anderen Datentypen in einen \pythonil{float}\nobreakdashes-Wert um. \pythonil{a}~könnte z.B.\ ein String sein.\uncover<15->{ \pythonil{float} erkennt die wissenschaftliche Notation und \pythonil{float(\"2.233e4\")} wird daher~\pythonil{22330.0}.%
}}%
%
\only<-17>{\item<16-> Natürlich geht das auch ohne wissenschaftliche Notation.\uncover<16->{ \pythonil{float(\"0.1123\")} wird zu~\pythonil{0.1123}.%
}}%
%
\only<-19>{\item<18-> Es werden auch die Konstanten für \inQuotes{zu groß für \pythonil{float}} und \inQuotes{keine Zahl} erkannt.\uncover<18->{ \pythonil{float(\"inf\")} wird zu~\pythonil{math.inf}.%
}}%
%
\only<-21>{\item<20-> Es werden auch die Konstanten für \inQuotes{zu groß für \pythonil{float}} und \inQuotes{keine Zahl} erkannt.\uncover<21->{ Und \pythonil{float(\"nan\")} wird zu~\pythonil{math.nan}.%
}}%
%
\only<-23>{\item<22-> Die Funktion~\pythonil{bool} wandelt einen Wert~\pythonil{a} eines anderen Datentypen in einen \pythonil{bool}\nobreakdashes-Wert um. \pythonil{a}~könnte z.B.\ ein String sein.\uncover<23->{ Aus \pythonil{bool(\"True\")} wird daher~\pythonil{True}.%
}}%
%
\only<-25>{\item<24-> Die Funktion~\pythonil{bool} wandelt einen Wert~\pythonil{a} eines anderen Datentypen in einen \pythonil{bool}\nobreakdashes-Wert um. \pythonil{a}~könnte z.B.\ ein String sein.\uncover<25->{ Und \pythonil{bool(\"False\")} wird zu~\pythonil{False}.%
}}%
%
\item<26-> Das war relativ einfach.%
%
\end{itemize}%
%
\locateGraphic{7}{width=0.8\paperwidth}{graphics/strBasic/strBasic03python3done}{0.1}{0.33}%
\locateGraphic{8}{width=0.8\paperwidth}{graphics/fromStr/fromStr01int1111}{0.1}{0.37}%
\locateGraphic{9}{width=0.8\paperwidth}{graphics/fromStr/fromStr02int1111done}{0.1}{0.37}%
\locateGraphic{10}{width=0.8\paperwidth}{graphics/fromStr/fromStr03int0x1111}{0.1}{0.37}%
\locateGraphic{11}{width=0.8\paperwidth}{graphics/fromStr/fromStr04int0x1111done}{0.1}{0.37}%
\locateGraphic{12}{width=0.8\paperwidth}{graphics/fromStr/fromStr05int0b1111}{0.1}{0.37}%
\locateGraphic{13}{width=0.8\paperwidth}{graphics/fromStr/fromStr06int0b1111done}{0.1}{0.37}%
\locateGraphic{14}{width=0.8\paperwidth}{graphics/fromStr/fromStr07float2d233e4}{0.1}{0.37}%
\locateGraphic{15}{width=0.8\paperwidth}{graphics/fromStr/fromStr08float2d233e4done}{0.1}{0.37}%
\locateGraphic{16}{width=0.8\paperwidth}{graphics/fromStr/fromStr09float0d1123}{0.1}{0.37}%
\locateGraphic{17}{width=0.8\paperwidth}{graphics/fromStr/fromStr10float0d1123done}{0.1}{0.37}%
\locateGraphic{18}{width=0.8\paperwidth}{graphics/fromStr/fromStr11floatInf}{0.1}{0.37}%
\locateGraphic{19}{width=0.8\paperwidth}{graphics/fromStr/fromStr12floatInfDone}{0.1}{0.37}%
\locateGraphic{20}{width=0.8\paperwidth}{graphics/fromStr/fromStr13floatNan}{0.1}{0.37}%
\locateGraphic{21}{width=0.8\paperwidth}{graphics/fromStr/fromStr14floatNanDone}{0.1}{0.37}%
\locateGraphic{22}{width=0.8\paperwidth}{graphics/fromStr/fromStr15boolTrue}{0.1}{0.37}%
\locateGraphic{23}{width=0.8\paperwidth}{graphics/fromStr/fromStr16boolTrueDone}{0.1}{0.37}%
\locateGraphic{24}{width=0.8\paperwidth}{graphics/fromStr/fromStr17boolFalse}{0.1}{0.37}%
\locateGraphic{25}{width=0.8\paperwidth}{graphics/fromStr/fromStr18boolFalseDone}{0.1}{0.37}%
\end{frame}%
%
%
\section{Strings Escaping}%
%
\begin{frame}[t]%
\frametitle{Strings Escaping}%
\begin{itemize}%
\only<-2,30->{%
\item Zeichenketten werden durch Anführungszeichen begrenzt.%
\item<2-> Was machen wir, wenn wir Anführungszeichen \alert{in} einer Zeichenkette haben wollen?%
}%
%
\only<-4>{\item<3-> Wenn wir unsere Strings durch doppelte Anführungszeichen~(\textil{\"}) begrenzen, dann können problemlos einfache Anführungszeichen~(\textil{\'}) einfügren.\uncover<4->{ Beachten Sie, dass wenn wir Strings mit \pythonil{print} ausgeben, die begrenzenden Anführungszeichen nicht in der Ausgabe enthalten sind.%
}}%
%
\only<-6>{\item<5-> Wenn wir doppelte Anführungszeichen~(\textil{\"}) brauchen, dann begrenzen wir unsere Strings eben mit einfachen Anführungszeichen~(\textil{\'}).\uncover<6->{ Beachten Sie, dass wenn wir Strings mit \pythonil{print} ausgeben, die begrenzenden Anführungszeichen nicht in der Ausgabe enthalten sind.%
}}%
%
\only<-11,30->{\only<-9,30->{\item<7-> Aber was machen wir, wenn wir \alert{beide} Arten von Anführungszeichen brauchen?\uncover<8->{ %
Dann können wir nicht auf \inQuotes{andere} Anführungszeichen zur Begrenzung der Strings ausweichen.}%
\uncover<9->{ %
\item<9-> Dann brauche wir sogenannte \glslink{escapeSequence}{Escape-Sequenzen}\cite{PSF:P3D:TPLR:ES}.}}%
\uncover<10->{ %
\item<10-> Wir können dem doppelten Anführungszeichen einfach in Backslash~(\inQuotes{\textbackslash}) voranstellen.\uncover<11->{ Dann wird es nicht mehr als String-Begrenzer interpretiert und verwandelt sich in ein normales doppeltes Anführungszeichen.%
}}%
}%
%
\only<-13>{\item<12-> So können wir bequem doppelte Anführungszeichen in Strings haben, die mit doppelten Anführungszeichen begrenzt sind.\uncover<13->{ Die Backslashes verschwinden dann in der Ausgabe.%
}}%
%
\only<-15>{\item<14-> So können wir bequem einfache Anführungszeichen in Strings haben, die mit einfache Anführungszeichen begrenzt sind.\uncover<13->{ Die Backslashes verschwinden dann in der Ausgabe.%
}}%
%
\only<-17>{\item<16-> Aber was machen wir, wenn wir ein Backslash~(\inQuotes{\textbackslash}) in einem String haben wollen?\uncover<17->{ Einfach noch ein Backslash davor schreiben. Aus dem doppelten Backslash wird dann ein einfaches Backslash.%
}}%
%
\only<-19>{\item<18-> Natürlich können wir beliebige viele \glslink{escapeSequence}{Escape-Sequenzen} in unseren Strings verwenden.\uncover<19->{ %
Der String \pythonil{\"\\\"\\\'\\\\\\\"\\\"\"} beginnt und endet mit \textil{\"}. Das erste Zeichen ist ein durch \textil{\\\"} escapedes~\textil{\"}. Darauf folget ein escapedes einfaches Anführungszeichen, ein escapedes Backslash, ein zweites escapedes doppeltes Anführungszeichen, und ein drittes escapedes doppeltes Anführungszeichen.%
}}%
%
\only<-21>{\item<20-> Die Sequenz~\textil{\\n} steht für \inQuotes{Newline}, also den Anfang einer neuen Zeile.\uncover<21->{ Somit können wir mehrzeilige Strings in einer einzigen Zeile schreiben.%
}}%
%
\only<-23>{\item<22-> Unter Windows kann man auch die Sequenz~\textil{\\r\\n} verwenden.\uncover<23->{ Muss man aber nicht, \python\ gibt auch mit~\textil{\\n} den richtigen Text aus.%
}}%
%
\only<-25>{\item<24-> Die Sequenz~\textil{\\t} steht für \inQuotes{Tabulator}, welcher oft equivalent zu mehreren (oft~4) Leerzeichen ausgegeben wird.\uncover<24->{ Das braucht man selten.%
}}%
%
\only<-29>{\item<26-> Wollen wir einen String über mehrere Zeilen hinweg schreiben, dann können wir ans Ende der Zeile einfach ein Backslash~(\inQuotes{\textbackslash}) schreiben.\uncover<27->{ Wenn wir dann \keys{\enter}~drücken, geht es in der nächsten Zeile einfach weiter, die im Interpreter dann mit drei Punkten beginnt.\uncover<28->{ So können wir den Text weiterschreiben.\uncover<29->{ Der String beinhaltet dann keinen Zeilenumbruch.}}%
}}%
%
\item<30-> \glslink{escapeSequence}{Escape-Sequenzen} sind also nützliche Werkzeuge, um Zeichen in Strings zu schreiben, die wir sonst nicht schreiben könnten.%
%
\end{itemize}%
%
\locateGraphic{2}{width=0.8\paperwidth}{graphics/strBasic/strBasic03python3done}{0.1}{0.33}%
\locateGraphic{3}{width=0.8\paperwidth}{graphics/escape/escape01singleQuote}{0.1}{0.37}%
\locateGraphic{4}{width=0.8\paperwidth}{graphics/escape/escape02singleQuoteDone}{0.1}{0.37}%
\locateGraphic{5}{width=0.8\paperwidth}{graphics/escape/escape03doubleQuote}{0.1}{0.37}%
\locateGraphic{6-7}{width=0.8\paperwidth}{graphics/escape/escape04doubleQuoteDone}{0.1}{0.37}%
\locateGraphic{10}{width=0.8\paperwidth}{graphics/escape/escape05bothQuotes}{0.1}{0.37}%
\locateGraphic{11}{width=0.8\paperwidth}{graphics/escape/escape06bothQuotesDone}{0.1}{0.37}%
\locateGraphic{12}{width=0.8\paperwidth}{graphics/escape/escape07doubleQuotes2}{0.1}{0.37}%
\locateGraphic{13}{width=0.8\paperwidth}{graphics/escape/escape08doubleQuotes2Done}{0.1}{0.37}%
\locateGraphic{14}{width=0.8\paperwidth}{graphics/escape/escape09singleQuotes2}{0.1}{0.37}%
\locateGraphic{15}{width=0.8\paperwidth}{graphics/escape/escape10singleQuotes2done}{0.1}{0.37}%
\locateGraphic{16}{width=0.8\paperwidth}{graphics/escape/escape11backslash}{0.1}{0.37}%
\locateGraphic{17}{width=0.8\paperwidth}{graphics/escape/escape12backslashDone}{0.1}{0.37}%
\locateGraphic{18}{width=0.8\paperwidth}{graphics/escape/escape13complex}{0.1}{0.37}%
\locateGraphic{19}{width=0.8\paperwidth}{graphics/escape/escape14complexDone}{0.1}{0.37}%
\locateGraphic{20}{width=0.8\paperwidth}{graphics/escape/escape15newline}{0.1}{0.37}%
\locateGraphic{21}{width=0.8\paperwidth}{graphics/escape/escape16newlineDone}{0.1}{0.37}%
\locateGraphic{22}{width=0.8\paperwidth}{graphics/escape/escape17winNewline}{0.1}{0.37}%
\locateGraphic{23}{width=0.8\paperwidth}{graphics/escape/escape18winNewlineDone}{0.1}{0.37}%
\locateGraphic{24}{width=0.8\paperwidth}{graphics/escape/escape19tab}{0.1}{0.37}%
\locateGraphic{25}{width=0.8\paperwidth}{graphics/escape/escape20tabDone}{0.1}{0.37}%
\locateGraphic{26}{width=0.8\paperwidth}{graphics/escape/escape21backslashNewline1}{0.1}{0.37}%
\locateGraphic{27}{width=0.8\paperwidth}{graphics/escape/escape22backslashNewline2}{0.1}{0.37}%
\locateGraphic{28}{width=0.8\paperwidth}{graphics/escape/escape23backslashNewline3}{0.1}{0.37}%
\locateGraphic{29}{width=0.8\paperwidth}{graphics/escape/escape24backslashNewlineDone}{0.1}{0.37}%
\end{frame}%
%
\section{Mehrzeilige Strings}%
%
\begin{frame}[t]%
\frametitle{Mehrzeilige Strings}%
\begin{itemize}%
\only<-3>{%
\item Wir können Strings erstellen, die über mehrere Zeilen gehen, in dem wir die \glslink{escapeSequence}{Escape\nobreakdashes-Sequenz}~\textil{\\n} einfügen.%
\item<2-> Wenn die Strings lang werden, dann kann das aber zu häßlichen sehr langen Kodezeilen führen.%
}%
%
\only<-4>{%
\item<3-> Für lange mehrzeilige Strings gibt es die Syntax mit dreifachen doppelten Anführungszeichen:~\pythonil{\"\"\"...\"\"\"}.%
}%
%
%
\only<-10>{\item<5-> Wir beginnen solche Strings immer mit \textil{\"\"\"} anstelle von \textil{\"}.\uncover<6->{ %
Machen wir einen Zeilenumbruch in so einem String\only<-6>{\dots}\uncover<7->{, %
dann können wir wieder auf der nächsten Zeile weiterschreiben\only<-7>{\dots}\uncover<8->{, %
die im Interpreter wieder mit drei Punkten anfängt.\uncover<9->{ %
Wir können so beliebig viele Zeilenumbrüche schreiben.\uncover<10->{ %
Diese sind dann Teil des Strings~(anders als bei \textil{\\} an Zeilenende!).%
}}}}}}%
%
\only<-16>{\item<11-> Diese mehrzeilige Syntax lässt sich mit \glslink{fstring}{f\nobreakdashes-Strings} kombinieren.\uncover<12->{ %
Solche mehrzeiligen \glslink{fstring}{f\nobreakdashes-Strings} sind von~\pythonil{f\"\"\"...\"\"\"} begrenzt.\uncover<13->{ %
Sie können beliebige Ausdrücke enthalten und normal \glslink{strinterpolation}{interpoliert}.%
}}}%
%
\item<17-> Somit haben wir nun auch eine Methode, große Textstücke als Strings in unsere Programme einzufügen.%
\item<18-> Das wird später bei der Dokumentation von Kode nützlich.%
%
\end{itemize}%
%
\locateGraphic{4}{width=0.8\paperwidth}{graphics/strBasic/strBasic03python3done}{0.1}{0.37}%
\locateGraphic{5}{width=0.8\paperwidth}{graphics/multiline/multiline01multiDQ1}{0.1}{0.37}%
\locateGraphic{6}{width=0.8\paperwidth}{graphics/multiline/multiline02multiDQ2}{0.1}{0.37}%
\locateGraphic{7}{width=0.8\paperwidth}{graphics/multiline/multiline03multiDQ3}{0.1}{0.37}%
\locateGraphic{8}{width=0.8\paperwidth}{graphics/multiline/multiline04multiDQ4}{0.1}{0.37}%
\locateGraphic{9}{width=0.8\paperwidth}{graphics/multiline/multiline05multiDQ5}{0.1}{0.37}%
\locateGraphic{10}{width=0.8\paperwidth}{graphics/multiline/multiline06multiDQdone}{0.1}{0.37}%
\locateGraphic{11}{width=0.8\paperwidth}{graphics/multiline/multiline07multiDQF1}{0.1}{0.37}%
\locateGraphic{12}{width=0.8\paperwidth}{graphics/multiline/multiline08multiDQF2}{0.1}{0.37}%
\locateGraphic{13}{width=0.8\paperwidth}{graphics/multiline/multiline09multiDQF3}{0.1}{0.37}%
\locateGraphic{14}{width=0.8\paperwidth}{graphics/multiline/multiline10multiDQF4}{0.1}{0.37}%
\locateGraphic{15}{width=0.8\paperwidth}{graphics/multiline/multiline11multiDQF5}{0.1}{0.37}%
\locateGraphic{16}{width=0.8\paperwidth}{graphics/multiline/multiline12multiDQFdone}{0.1}{0.37}%
%
\uncover<19->{%
\bestPractice{longstrDoubleQuote}{Wenn wir mehrzeilige Strings definieren, dann sollte immer doppelte Anführungszeichen verwendet werden und nicht einfache~(natürlich immer in dreifacher Ausfertigung).\cite{PEP257,PEP8}}%
}%
\end{frame}%
%
\section{Unicode}%
%
\begin{frame}%
\frametitle{Texte als Zahlen}%
\begin{itemize}%
\item Der Speicher unserer Computer speichert Gruppen von Bits in fester Größe, sagen wir, Bytes zu jeweils 8~Bit.%
\item<2-> Diese Bytes werden oft als Ganzzahlen betrachtet.%
\item<3-> Während \python\ \inQuotes{beliebig-große} \pythonils{int} anbietet, die aus entsprechend vielen Bytes bestehen, nutzen wir in anderen Programmiersprachen oftmals jeweils 8~Bytes, um einen Integer darzustellen.
\item<4-> Der Datentyp \pythonil{float} ist in \python\ ebenfalls 8~Bytes groß -- wie diese interpretiert werden haben Sie schon gelernt.%
\item<5-> Aber wie funktioniert das mit Text?%
\item<6-> In dem jedes Zeichen eines Textes als eine Zahl dargestellt wird.%
\item<7-> Ein \pythonil{str} ist dann nichts als eine Sequenz von Zahlen.%
\item<8-> Das System weiß, welches Zeichen zu welcher Zahl gehört.
\end{itemize}%
\end{frame}%
%
\begin{frame}%
\frametitle{Zeichen als Zahlen}%
\begin{itemize}%
\item Historisch gesehen gibt es viele \inQuotes{Zeichen-zu-Zahlen} Mappings.%
\item<2-> Die bekannteste solche Zuordnung ist ASCII\cite{ASA1963ASCII,USAS1967USCFII}, welche sieben Bits pro Zeichen verwendete.%
\item<3-> Darum konnte ASCII auch nur Lateinische Buchstaben, Satzzeichen, Ziffern, und ein Kontrollzeichen~(wie \textil{\\t} und \textil{\\n}\dots) darstellen.%
\item<4-> Ihnen ist vielleicht aufgefallen, dass verschiedene Sprachen verschiedene Zeichen verwenden {\dots} und zwar mehr als insgesamt~$2^7=128$.%
\item<5-> Deshalb gibt es verschiedene solche Zuordnungen.%
\item<6-> In China existieren verschiedene, auf Chinesische Zeichen spezialisierte, Zuordnungen, z.B.\ das historische GB~2312\cite{GBT23121980PROCNSG21CCECSFIE}, GBK\cite{GBK1995CICSNSE}, und das neuere GB~18030\cite{GB180302022ITCCCS}.%
\item<7-> Die große Mehrheit der Systeme heute~(und \python) verwendet einen gemeinsamen Standard für \alert{alle} Sprachen: \glsFull{unicode}\cite{TUC2023U1510,TUC2023U151ACS,ISOIEC106462020ITUCCSU}.%
\end{itemize}%
\end{frame}%
%
\begin{frame}[t]%
\frametitle{Unicode Beispiel}%
\only<-1,21->{%
\begin{itemize}%
\only<-23,27->{%
\item Jedem Zeichen ist eine Zahl zugeordnet.\only<-1>{ Gucken wir uns das mal an.}%
\item<21-> \inQuotes{你}~entspricht der Hexadezimalzahl~4f60, \inQuotes{好}~ist durch die Hexadezimalzahl~597d repräsentiert, und der Voll\nobreakdashes-weite Satzpunkt~\inQuotes{。} entspricht der Hexadezimalzahl~3002.%
\item<22-> Alle Zeichen können als so-genanntes Unicode-Escape, also \textil{\\u}~gefolgt von der entsprechenden Hexadezimalzahl geschrieben Werden.}%
\only<-24>{\item<23->{ Gucken wir uns das mal an.}}%
%
\item<25-> Wir verwenden die Unicode-Escapes \textil{\\u4f60}, \textil{\\u597d}, und \textil{\\u3002} für die drei Zeichen~\inQuotes{你好。}.\uncover<26->{ %
Und tatsächlich wird der korrekte Text gedruckt.%
}%
%
\item<27-> Damit haben Sie nun also auch eine Vorstellung, wie Text im Speicher des Computers dargstellt wird.%
%
\end{itemize}%
}%
\locateGraphic{2}{width=0.42\paperwidth}{graphics/unicodeCharacterTableSubset/unicodeCharacterTableSubset01}{0.29}{0.1}%
\locateGraphic{3}{width=0.42\paperwidth}{graphics/unicodeCharacterTableSubset/unicodeCharacterTableSubset02}{0.29}{0.1}%
\locateGraphic{4}{width=0.42\paperwidth}{graphics/unicodeCharacterTableSubset/unicodeCharacterTableSubset03}{0.29}{0.1}%
\locateGraphic{5}{width=0.42\paperwidth}{graphics/unicodeCharacterTableSubset/unicodeCharacterTableSubset04}{0.29}{0.1}%
\locateGraphic{6}{width=0.42\paperwidth}{graphics/unicodeCharacterTableSubset/unicodeCharacterTableSubset05}{0.29}{0.1}%
\locateGraphic{7}{width=0.42\paperwidth}{graphics/unicodeCharacterTableSubset/unicodeCharacterTableSubset06}{0.29}{0.1}%
\locateGraphic{8}{width=0.42\paperwidth}{graphics/unicodeCharacterTableSubset/unicodeCharacterTableSubset07}{0.29}{0.1}%
\locateGraphic{9}{width=0.42\paperwidth}{graphics/unicodeCharacterTableSubset/unicodeCharacterTableSubset08}{0.29}{0.1}%
\locateGraphic{10}{width=0.42\paperwidth}{graphics/unicodeCharacterTableSubset/unicodeCharacterTableSubset09}{0.29}{0.1}%
\locateGraphic{11}{width=0.42\paperwidth}{graphics/unicodeCharacterTableSubset/unicodeCharacterTableSubset10}{0.29}{0.1}%
\locateGraphic{12}{width=0.42\paperwidth}{graphics/unicodeCharacterTableSubset/unicodeCharacterTableSubset11}{0.29}{0.1}%
\locateGraphic{13}{width=0.42\paperwidth}{graphics/unicodeCharacterTableSubset/unicodeCharacterTableSubset12}{0.29}{0.1}%
\locateGraphic{14}{width=0.42\paperwidth}{graphics/unicodeCharacterTableSubset/unicodeCharacterTableSubset13}{0.29}{0.1}%
\locateGraphic{15}{width=0.42\paperwidth}{graphics/unicodeCharacterTableSubset/unicodeCharacterTableSubset14}{0.29}{0.1}%
\locateGraphic{16}{width=0.42\paperwidth}{graphics/unicodeCharacterTableSubset/unicodeCharacterTableSubset15}{0.29}{0.1}%
\locateGraphic{17}{width=0.42\paperwidth}{graphics/unicodeCharacterTableSubset/unicodeCharacterTableSubset16}{0.29}{0.1}%
\locateGraphic{18}{width=0.42\paperwidth}{graphics/unicodeCharacterTableSubset/unicodeCharacterTableSubset17}{0.29}{0.1}%
\locateGraphic{19}{width=0.42\paperwidth}{graphics/unicodeCharacterTableSubset/unicodeCharacterTableSubset18}{0.29}{0.1}%
\locateGraphic{20}{width=0.42\paperwidth}{graphics/unicodeCharacterTableSubset/unicodeCharacterTableSubset19}{0.29}{0.1}%
%
\locateGraphic{24}{width=0.8\paperwidth}{graphics/strBasic/strBasic03python3done}{0.1}{0.37}%
\locateGraphic{25}{width=0.8\paperwidth}{graphics/unicodeEscape/unicodeEscape01print}{0.1}{0.37}%
\locateGraphic{26}{width=0.8\paperwidth}{graphics/unicodeEscape/unicodeEscape02printDone}{0.1}{0.37}%
\end{frame}%
%
\section{Zusammenfassung}%
%
\begin{frame}%
\frametitle{Zusammenfassung}%
\begin{itemize}%
\item Zeichenketten, Strings, also Instanzen der Klasse~\pythonil{str} sind wichtig.%
\item<2-> Oftmals bekommen wir Daten in Form von solchen Zeichenketten als Input für unsere Programme.%
\item<3-> Wir wandeln sie dann in andere Datentypen, with \pythonil{int} oder \pythonil{float} um, um mit ihnen zu arbeiten.%
\item<4-> Danach geben wir die Ergebnisse unserer Berechnungen wieder als Text aus. Dabei können wir z.B.\ f\nobreakdashes-Strings nutzen, um andere Datentypen elegant als Strings zu formatieren.%
\item<5-> Manchmal wollen wir auch mit Strings direkt arbeiten, z.B.\ Teile von ihnen extrahieren oder Groß-\ in Kleinbuchstaben umwandeln.%
\item<6-> Wir können Sonderzeichen durch Escape-Sequenzen in Strings einführen und mehrzeilige Strings erstellen.%
\item<7-> Intern werden die Zeichen einer Zeichenkette durch Zahlen repräsentiert, wobei die Zuordnung von Zahlen zu Zeichen meist auf \pgls{unicode} basiert.%
\item<8-> Das war schon recht viel. Hat aber auch Spaß gemacht.%
\end{itemize}%
\end{frame}%
%
\endPresentation%
\end{document}%%
\endinput%
%

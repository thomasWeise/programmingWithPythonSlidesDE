\pdfminorversion=7%
\documentclass[aspectratio=169,mathserif,notheorems]{beamer}%
%
\xdef\bookbaseDir{../../bookbase}%
\xdef\sharedDir{../../shared}%
\RequirePackage{\bookbaseDir/styles/slides}%
\RequirePackage{\sharedDir/styles/styles}%
\toggleToGerman%
%
\subtitle{4.~PyCharm Installieren}%
%
\begin{document}%
%
\startPresentation%
%
\section{Einleitung}%
%
\begin{frame}[t]%
\frametitle{Einleitung}%
\begin{itemize}%
\item Nur die Programmiersprache \python\ auf unserem System zu haben ist nicht genug.%
\item<2-> OK, technisch gesehen, ist es genug um \python-Programme auszuführen.%
\item<3-> Aber um vernünftig und effizient Software zu entwickeln, braucht man schon mehr.%
\item<4-> Wir wollen ja\only<5->{ \alert<5>{\textbf{nicht}}} mit dem \microsoftWindows\ NotePad programmieren\dots%
\item<6-> Wir brauchen ein \glsreset{ide}\pgls{ide}, ein Programm mit dem man mehrere der notwendigen Aufgaben in der Softwareentwicklung durch eine bequeme Benutzeroberfläche erledigen kann.%
\item<7-> Für \python\ empfehlen wir die \pycharm\cite{VHN2023HOADWP,Y2022PPADT,W2024PME} \pgls{ide}, deren Community Edition kostenlos ist.%
\item<8-> Die Installationsanweisungen für \pycharm\ finden sich unter \url{https://www.jetbrains.com/help/pycharm/installation-guide.html}.%
\end{itemize}%
\locateGraphic{4}{width=0.4\paperwidth}{graphics/usingNotepadAsEditorNoCross}{0.3}{0.45}%
\locateGraphic{5}{width=0.4\paperwidth}{graphics/usingNotepadAsEditor.pdf}{0.3}{0.45}%
\end{frame}%
%

\begin{frame}[t]%
\frametitle{PyCharm unter Ubuntu Linux installieren}%
\begin{itemize}%
\item Nachdem wir die Installationsanweisungen für die \pycharm\ Community Edition fertig gemacht haben, wurde angekündigt, dass diese von einer \inQuotes{unified edition} ersetzt wird.%
\item<2-> Die folgenden Anweisungen sind daher wahrscheinlich etwas veraltet, sollten aber dennoch ausreichend instruktiv sein.%
\end{itemize}%
\locateGraphicTB{}{width=0.5\paperwidth}{graphics/unifiedPyCharm}{0.25}{0.375}%
\end{frame}%
%
\section{PyCharm unter Ubuntu Linux installieren}%
%
\begin{frame}[t]%
\frametitle{PyCharm unter Ubuntu Linux installieren}%
\begin{itemize}%
\only<-2>{%
\item \pycharm\ steht unter \ubuntu\ \linux\cite{C2025SD,J2024PCPCE} als Snap-Package zur Verfügung.%
\item<2-> Der Installationsprozess ist sehr einfach.%
}%
%
\only<-4>{\item<3-> Zuerst öffnen wir ein Terminal in wir \ubuntuTerminal\ drücken.%
\item<4-> Dann schreiben wir das Kommando \bashil{sudo snap install pycharm-community --classic} und drücken~\keys{\enter}.}%
\only<-5>{\item<5-> Zur Installation brauchen wir die \glsreset{sudo}\pgls{sudo}\nobreakdashes-Privilegien und müssen daher das \pgls{sudo}\nobreakdashes-Passwort eingeben.}%
\only<-9>{\item<6-> Die Installation verläuft automatisch.}%
\only<-11>{\item<10-> Nachdem sie abgeschlossen ist, können Sie den \ubuntu\ Launcher durch Druck auf \keys{\OSwin} öffnen und im Launcher-Fenster \bashil{pycharm} eingeben.\uncover<11->{ Klicken Sie auf das \pycharm-Symbol.}}
\only<-12>{\item<12-> Der \pycharm\ Startbildschirm erscheint.}%
\only<-13>{\item<13-> Gegebenenfalls müssen Sie nun Lizenzbedingungen entweder zustimmen oder diese ablehnen.}%
\item<14-> Jetzt läuft \pycharm!%
\end{itemize}%
%
\locateGraphic{3-4}{width=0.8\paperwidth}{graphics/installingPyCharmUbuntu/installingPyCharmUbuntu01snapInstall}{0.1}{0.4}%
\locateGraphic{5-6}{width=0.8\paperwidth}{graphics/installingPyCharmUbuntu/installingPyCharmUbuntu02sudo}{0.1}{0.4}%
\locateGraphic{7}{width=0.8\paperwidth}{graphics/installingPyCharmUbuntu/installingPyCharmUbuntu03snapInstall}{0.1}{0.4}%
\locateGraphic{8}{width=0.8\paperwidth}{graphics/installingPyCharmUbuntu/installingPyCharmUbuntu04snapInstall}{0.1}{0.4}%
\locateGraphic{9}{width=0.8\paperwidth}{graphics/installingPyCharmUbuntu/installingPyCharmUbuntu05snapInstallFinished}{0.1}{0.4}%
\locateGraphicTB{10-11}{width=0.6\paperwidth}{graphics/installingPyCharmUbuntu/installingPyCharmUbuntu06launcher.pdf}{0.2}{0.3}%
\locateGraphicTB{12-13}{width=0.6\paperwidth}{graphics/installingPyCharmUbuntu/installingPyCharmUbuntu07welcome}{0.2}{0.3}%
\locateGraphicTB{14}{width=0.5\paperwidth}{graphics/installingPyCharmUbuntu/installingPyCharmUbuntu08pycharm}{0.25}{0.22}%
\end{frame}%
%
\section{PyCharm unter Microsoft Windows installieren}%
%
\begin{frame}[t]%
\frametitle{PyCharm unter Microsoft Windows installieren}%
\begin{itemize}%
\only<-1>{\item Nun wollen wir \pycharm\ unter \microsoftWindows\ herunterladen und installieren.}%
\only<-2>{\item<2-> Die Download-Webseite für \pycharm\ ist \url{https://www.jetbrains.com/pycharm/download} for \pycharm.}%
\only<-3>{\item<3-> Wir klicken auf \menu{.exe (Windows)} um den \pycharm\ Community Edition-Download zu starten.}%
\only<-4>{\item<4-> In dem aufklappenden PopUp-Menü klicken wie ebenfalls auf \menu{.exe (Windows)}.}%
\only<-5>{\item<5-> Der Download beginnt.}%
\only<-6>{\item<6-> Nach dem der Download abgeschlossen ist, führen wir das heruntergeladene Programm aus -- z.B.\ durch Klick auf \menu{Open file}.}%
\only<-7>{\item<7-> Der Installer startet.}%
\only<-8>{\item<8-> Der Installer tut und macht.}%
\only<-9>{\item<9-> Da wir \pycharm\ installieren wollen, klicken wir auf \menu{Yes} und erlauben dem Programm, unser System zu verändern.}%
\only<-10>{\item<10-> Wir sind im Willkommensbildschirm des Installers angekommen. Wir klicken \menu{Next}.}%
\only<-11>{\item<11-> Wir können den Installationsordner auswählen (oder einfach auf der Standareinstellung lassen) und klicken \menu{Next}.}%
\only<-12>{\item<12-> Wir lassen alle Optionen auf den Standardeinstellungen und klicken \menu{Next}.}%
\only<-13>{\item<13-> Wir lassen alle Optionen auf den Standardeinstellungen und klicken \menu{Install}.}%
\only<-14>{\item<14-> Die Installation beginnt.}%
\only<-15>{\item<15-> Die Installation ist abgeschlossen. Wir wählen \inQuotes{Run PyCharm Community Edition} aus und klick \menu{Finish}.}%
\only<-16>{\item<16-> Der Willkommensbildschirm von \pycharm.}%
\only<-17>{\item<17-> Wir können entscheiden, ob wir den Nutzungsbedingungen zustimmen und klicken auf \menu{Continue}.}%
\only<-18>{\item<18-> Wir können entscheiden, ob wir Nutzerstatistiken an JetBrains schicken wollen. Ich entscheide mich für \menu{Don't Send}.}%
\item<19-> \pycharm\ läuft!%
%
\end{itemize}%
%
\locateGraphicTB{2}{width=0.7\paperwidth}{graphics/installingPyCharmWindows/installingPyCharmWindows01download.pdf}{0.15}{0.26}%
\locateGraphicTB{3}{width=0.7\paperwidth}{graphics/installingPyCharmWindows/installingPyCharmWindows02download.pdf}{0.15}{0.26}%
\locateGraphicTB{4}{width=0.7\paperwidth}{graphics/installingPyCharmWindows/installingPyCharmWindows03download.pdf}{0.15}{0.26}%
\locateGraphicTB{5}{width=0.7\paperwidth}{graphics/installingPyCharmWindows/installingPyCharmWindows04download}{0.15}{0.26}%
\locateGraphicTB{6}{width=0.7\paperwidth}{graphics/installingPyCharmWindows/installingPyCharmWindows05runInstaller.pdf}{0.15}{0.26}%
\locateGraphicTB{7}{width=0.7\paperwidth}{graphics/installingPyCharmWindows/installingPyCharmWindows06runInstaller}{0.15}{0.26}%
\locateGraphicTB{8}{width=0.7\paperwidth}{graphics/installingPyCharmWindows/installingPyCharmWindows07runInstaller}{0.15}{0.26}%
\locateGraphicTB{9}{width=0.5\paperwidth}{graphics/installingPyCharmWindows/installingPyCharmWindows08runInstaller.pdf}{0.25}{0.28}%
\locateGraphicTB{10}{width=0.5\paperwidth}{graphics/installingPyCharmWindows/installingPyCharmWindows09installation.pdf}{0.25}{0.25}%
\locateGraphicTB{11}{width=0.5\paperwidth}{graphics/installingPyCharmWindows/installingPyCharmWindows10installation.pdf}{0.25}{0.25}%
\locateGraphicTB{12}{width=0.5\paperwidth}{graphics/installingPyCharmWindows/installingPyCharmWindows11installation.pdf}{0.25}{0.25}%
\locateGraphicTB{13}{width=0.5\paperwidth}{graphics/installingPyCharmWindows/installingPyCharmWindows12installation.pdf}{0.25}{0.25}%
\locateGraphicTB{14}{width=0.5\paperwidth}{graphics/installingPyCharmWindows/installingPyCharmWindows13installation}{0.25}{0.25}%
\locateGraphicTB{15}{width=0.5\paperwidth}{graphics/installingPyCharmWindows/installingPyCharmWindows14installation.pdf}{0.25}{0.25}%
\locateGraphicTB{16}{width=0.5\paperwidth}{graphics/installingPyCharmWindows/installingPyCharmWindows15running}{0.25}{0.3}%
\locateGraphicTB{17}{width=0.5\paperwidth}{graphics/installingPyCharmWindows/installingPyCharmWindows16running.pdf}{0.25}{0.3}%
\locateGraphicTB{18}{width=0.5\paperwidth}{graphics/installingPyCharmWindows/installingPyCharmWindows17running.pdf}{0.25}{0.3}%
\locateGraphicTB{19}{width=0.5\paperwidth}{graphics/installingPyCharmWindows/installingPyCharmWindows18running}{0.25}{0.25}%
\end{frame}%
%
\section{Zusammenfassung}%
%
\begin{frame}\frametitle{Zusammenfassung}%
\begin{itemize}%
\item Nun haben wir \pycharm\ installiert.%
\item<2-> Damit haben wir eine komfortables \glsreset{ide}\pgls{ide} für die Programmiersprache \python.%
\item<3-> Cool.%
\end{itemize}%
\end{frame}%
%
\endPresentation%
\end{document}%%
\endinput%
%
